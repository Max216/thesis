\documentclass[article,type=msc,colorback,accentcolor=tud9c]{tudthesis}
\usepackage[english]{babel}
\usepackage{titlesec}
\usepackage{natbib}
\bibliographystyle{acl_natbib}
%\setcitestyle{authoryear,open={((},close={))}}
\setcounter{secnumdepth}{4}
\usepackage{acronym}
\usepackage{multirow}
\usepackage{booktabs,tabularx}
\usepackage{placeins}
\usepackage{graphicx}
\usepackage{etoolbox}
\usepackage{setspace} 
\AtBeginEnvironment{quote}{\singlespacing\normalfont}
\usepackage{amsmath}
\usepackage{hyperref}
\usepackage{wrapfig}

\titleformat{\paragraph}
{\normalfont\normalsize\bfseries}{\theparagraph}{1em}{}
\titlespacing*{\paragraph}
{0pt}{3.25ex plus 1ex minus .2ex}{1.5ex plus .2ex}
\newcommand{\getmydate}{%
  \ifcase\month%
    \or Januar\or Februar\or M\"arz%
    \or April\or Mai\or Juni\or Juli%
    \or August\or September\or Oktober%
    \or November\or Dezember%
  \fi\ \number\year%
}

\newcommand{\mx}[1]{\textcolor{red}{max: #1}}

\begin{document}
  \thesistitle{Understanding and Improving Neural Models for Natural Language Interference}%
    {Verständnis und Verbesserung Neuronaler Modelle für Natural languge inference}
  \author{Max Glockner}
  \referee{Prof. Dr. Iryna Gurevych}{Dr. Yoav Goldberg}[Dr. Andreas R\"ucké]
  \department{Fachbereich Physik}
  \group{Institut f"ur Angewandte Festkernphysik\\Speerspitze der Elite}
  \dateofexam{\today}{\today}
  \tuprints{12345}{1234}
  \makethesistitle
  \affidavit{J. Walker}

\section*{Abstract}
\addtocounter{section}{0}
TODO in English...
\section*{Zusammenfassung}
\addtocounter{section}{0}
TODO in Deutsch...

\section*{List of abbreviations}
\addtocounter{section}{0}

\begin{acronym}[Bash]
 \acro{biLSTM}{bidirectional Long-Short-Term Memory Network}
 \acro{BoW}{Bag of Words}
 \acro{ESIM}{Enhanced Sequential Inference Model}
 \acro{IE}{Information Extraction}
 \acro{IR}{Information Retrieval}
 \acro{KIM}{Knowledge-based Inference Model}
 \acro{LSTM}{Long-Short-Term-Memory}
 \acro{MLP}{Multi Layer Perceptron}
 \acro{MultiNLI}{MultiGenre Natural Language Inference Corpus}
 \acro{NLI}{Natural Language Inference}
 \acro{NLP}{Natural Language Processing}
 \acro{NLU}{Natural Language Understanding}
 \acro{POS}{Part of Speech}
 \acro{OANC}{Open American National Corpus}
 \acro{QA}{Question Answering}
 \acro{RNN}{Recurrent Neural Network}
 \acro{RTE}{Recognizing Textual Entailment}
 \acro{SD}{Standard Deviation}
 \acro{SICK}{Sentences Involving Compositional Knowledge}
 \acro{SNLI}{The Stanford Natural Language Inference Corpus}
 \acro{WSD}{Word Sense Disambiguation}
 \acro{YAGO}{Yet Another Great Ontology}


\end{acronym}
\tableofcontents
\newpage\cleardoublepage
\section{Introduction}
In recent years neural networks again gained a lot of popularity for many machine learning tasks, including the field of \ac{NLP}. While previous generation solutions heavily depended on handcrafted features, these models are capable of learning meaningful feature representations automatically \citep{bengio2013representation}, thus avoiding time consuming process of feature-engineering. For the most part, neural networks solely rely on distributed word representations, also known refered to as word-embeddings, like word2vec \citep{mikolov2013distributed} or GloVe \citep{pennington2014glove} and typically learn fixed-length dense vector representations for the input text. While they provie strong generalization capabilities, they fail to capture simple world-knowledge \citep{celikyilmaz2010enriching} and even have trouble differentiationg between mutually exclusive words, if they generally are used in similar contexts \citep{vulic2017morph}. As opposed to that, traditional approaches execcively made usage of lexical resources containing relational and factual information about words and entities, thus providing a huge amount of ready-to-sue knowledge bases. Intuitivly, combining both worlds by integrating existing knowldge bases into neural networks should even further improve these models, due to a more sophisticated \ac{NLU}. This is analogue tothe way humans understand text, by having a solid understanding of the world, that influences the subjective interpretation of every word within a sentence. Given the sentence ``The official language in the USA is English.'' an average human can conclude that the official language of \textit{New York} also is English, knowing that \textit{New York} is within the \textit{USA}.
\newline

To improve the \ac{NLU} of neural models for \ac{NLP} we address the mentioned problems by analysing the sentence representations of a state-of-the-art model and identifying knowledge that is captured or not captured using state-of-the-art strategies without external resources. We show those state-of-the-art models are limited in their generalization ability and fail to capture simple inferences. To overcome this problem, we evaluate how additional knowledge from exxternal resources could be inferred neural networks. While our aim is to provide generally applicable results, we base our experiments on the task of \ac{NLI} \citep{bowman2015large}, also known as \ac{RTE} \citep{dagan2006pascal}. As this is known to be a fundamental task for \ac{NLU} \citep{maccartney2007natural}, insights gained here can improve other tasks of \ac{NLP} that indirectly depend on it. 

\paragraph*{Structure of the Thesis}
While we explain relevant techniques and concepts, we expect the reader to have a basic understanding of common machine-learning practices, neural networks, including basic network architectures like \ac{LSTM} or \ac{RNN}, and \ac{NLP} in general. This thesis is structured in the following manner:
\begin{itemize}
\item Section §\ref{sec:basics} is used to give definitions for \ac{NLI} and relevant word-relations. We further give a detailed description about the architecture and traning of the state-of-the-art model, that we use throughout all our experiments.
\item In section §\ref{sec:related_work} we introduce recently publish relevant datasets for \ac{NLI}and discuss several strategies proved to be successful. In addition we show a selectionn of lexical resources that contain relevant information to improve the \ac{NLU} of neural models and various strategies that have been applied to integrate them.
\item We analyse how the information of a natural language text is encoded within the sentence representation of a neural model and give insights on how the model uses it in section §\ref{sec:understanding}.
\item We derive a new testset from a major dataset for \ac{NLI}, demonstrating the poor generalization abilities of state-of-the-art models in section §\ref{sec:additional_snli_set}.
\item Based on the new data we evaluate whether external resources are helpful for the task using advanced embeddings and multitask-learning.
\end{itemize}\newpage\cleardoublepage
\section{Theoretical Background}\label{sec:basics}
This section gives an overview of the task that is used within this work (§\ref{sec:basics_nli}) and relevant datasets regarding this task (§\ref{sec:basics_datasets}).
\subsection{Natural Language Inference}\label{sec:basics_nli}
\ac{NLI} \citep{bowman2015large} deals with the problem to identify, whether one piece of natural text, namely the \textit{hypothesis}, can be inferred from another piece of text, namely the \textit{premise}. The hypothesis $h$ is said to be entailed by the premise $p$ if a human reader would conclude that the hypothesis is true, given the fact that the premise is true. Therefore it differes from strict logical inference. While in \ac{NLI} a high plausability for the premise to imply the hyothesis, based on the human judgement, is sufficient, the latter one strives to achieve certainity \citep{dagan2009recognizing}. \ac{NLI} essentially breaks down to an alignment problem \citep{maccartney2008phrase}. Given the sentence pair
\begin{center}
\begin{tabular}{rl}
\textbf{Premise:} & \textit{Donald Trump is eating his cheeseburger in his bedroom.} \\
\textbf{Hypothesis:} & \textit{The president of the United States is snacking a cheeseburger in the White House.} 
\end{tabular}
\end{center}
the model is required to correctly align \textit{Donald Trump} with \textit{The president of the United States}, \textit{eating} with \textit{snacking} and have information that his \textit{bedroom} is within the \textit{White House}. Here it can be seen, how the system would not only need to cope with different ways of expressing the same meaning, due to the nature of language, but also is required to access and process factual information, that is commonly known to an average human. 
\newline
\subsubsection*{Relatedness to other NLP tasks}
While \ac{NLI} clearly is central to reasoning capabilities, it is very fundamental and applicable to a large variety of \ac{NLP} as the ability to recognize textual entailment is a fundamental and necessary problem towards real \ac{NLU} \citep{maccartney2007natural,bos2005recognising}. Many \ac{NLP} applications such as \ac{QA}, Summarization \ac{IE} implicitly depend on this ability, as the huge variability of possible expressions for the same meaning it is a core phenomen of natural language \citep{dagan2009recognizing}. All three tasks require the model to infer that the target meaning of interest can be inferred from any other variant of textual expression. For \ac{QA} it is the identification of a correct answer, for summarization, the complete summary needs to be implied by the original text and similarily, while redundant sentences expressing the same meaning should be ommited. Similarily \ac{IE}, especially if using multiple documents, needs to infer, whether two variants of text contain the same information. Even simple paraphrasing can be broken down to a lexical inference problem with mutual entailment between $p$ and $h$. As end applications for \ac{NLP} in addition to \ac{NLU} need to solve another complicated machine-learning task it is hard to compare and directly improve their \ac{NLU} capabilities. Thus, one of the main purposes of \ac{NLI}, being a very basic problem towards \ac{NLU}, serves a a benchmark  with any improvements helping a large variety of high level tasks \citep{williams2017broad,cooper1996using,bos2005recognising,dagan2006pascal}. 

\subsection{Lexical Semantic Relations}\label{sec:word_relations}
\mx{TODO: lexical inference \citep{shwartz2015learning} and being the semantic relation between two terms (bi/unidirectional)}
Lexical relations describe the relationship between words\footnote{While there is no single definition for \textit{word}, we refer assume a single word to have the same surface form and the same lemma.} whereas \textit{Lexical Semantic Relations} specifically indicate relations referring to the meaning of the word. \citep{murphy2003semantic} and have shown to be helpful for detecting lexical inferences \citep{dagan2009recognizing}. We define the following relations based on those of \cite{Jurafsky2008May}. One key characteristic of natural language is ambiguity, tht is also present in lexical semantics as words may have several meanings or \textit{senses}\footnote{The phenomen of words having multiple senses is called \textit{homonymy}, if both senses show indicate no relation but share the surface form like ``bank'' (financial institution) and ``bank'' (sloping mound). If those senses are semantically related like ``milk'' (take milk from female mammals) and ``milk'' (like cow's milk), the relationship is called \textit{polysemy} \citep{Jurafsky2008May}}. To deal with this phenomene, lexical semantic relations are defined between senses rather than words. For the sake of simplicity for the most part we follow a naive approach in the following chapters of assuming the most dominant sense of a word, when referring to it. Specifically we define \textit{Synonymy}, \textit{Antonomy}, \textit{Hypernomy} and \textit{Holonomy}, the latter two relations are visualized\footnote{This is for illustration purposes only and we only added some relevant relations between the entities, more relation are possible. For instance, the holonym relationship would of course hold between \textit{head} and any other \textit{animal}.} in Figure \ref{fig:lexical_resources}.
\begin{figure}[tph!]
\centering
	\includegraphics[totalheight=7cm]{fig/lexical_relations.png}
	\caption{A sample ontology of animals to illustrate the lexical relations \textit{Hypernomy} and \textit{Holonymy}.}
	\label{fig:lexical_resources}
\end{figure}
\subsubsection{Synonymy and antonomy}
Synonomy is a symmetric relationship between two senses or two words. Two senses of two different words are said to be synonyms, if they have the same or nearly the same meaning. Synonymy between words is holds, if one word can be replaced by the other word in any sentence  without changing the meaning of the sentence. True synonyms are rare, as most words at least have subtle differences in their meaning or are used within different contexts. We thus follow common practice and loosen the strict definition by refering to synonmys if they have approximately similar meanings. Like synonomy, antomy is a symmetric relationship between senses, however having the opposite meaning, which might be caused by a binary opposition like ``opened/closed'', by different ends on some scale like ``hot/cold'' or by directional change like ``upwards/downwards''. Since antonyms semantically are identical in all other aspects with synonyms, these relations are hard to distinguish from each other automatically.

\subsubsection{Hypernomy}
Hypernomy (or Hyponomy) is an asymmetric relation between two senses and also referred to as the \textbf{is-a} relation. The more specific sense (e.g. \textit{bee}) is called a hyponym of the more general sense (e.g. \textit{insect}), which is called hypernym. \cite{Jurafsky2008May} give a formal definition for Hyponomy in terms of entailment: 
\begin{quotation}\noindent
``[...] a sense $A$ is a hyponym of a sense $B$ if everything that is $A$ is also $B$ and hence being an $A$ entails being a $B$, or $\forall x$ $A(x) \Rightarrow B(x)$.'' \citep{Jurafsky2008May}
\end{quotation}
Hypernomy is in most casses transitiv, thus if a \textit{cow} is hyponym of \textit{mammal} and \textit{mammal} is a hyponym of \textit{animal}, \textit{cow} is also a hyponym of \textit{animal}. an important phenomen for this thesis are two words, sharing a close hypernyms. In Figure \ref{fig:lexical_resources}, \textit{bee} and \textit{butterfly} share the hypernym \textit{insect}, we refer to them as co-hyponyms.

\subsubsection{Holonomy}
Holonomy or Meronomy refers to the \textbf{part-whole} relation. In the illustration of Figure \ref{fig:lexical_resources}, the \textit{wing} is a part of a \textit{bird} and a \textit{bird} is a part of a \textit{flock}. We say that a \textit{bird} is a meronym of \textit{flock}, while \textit{flock} is the holonym of \textit{bird}. As opposed to Hypernyomy, this asymmetric relation is not automatically transitiv. While a \textit{flock}\footnote{This is an example for polysemy, as \textit{flock} may refer to a group of birds, but also to a group of e.g. sheep. In this case we assume the sense of a group of birds.} obviously consists of several birds, In this case \textit{birds} is generally not replacable with \textit{heads}. 

\subsection{Shortcut-Stacked-Encoder and Residual Encoder}\label{sec:residual_encoder_def}
We conducted most of our experiments with the Shortcut-Stacked-Encoder \citep{nie2017shortcut} and the recently adapted version to the Residual Encoder for \ac{NLI}. They achieve state-of-the-art results\footnote{Considering models for SNLI without inter-sentence attention.} for two large datasets\footnote{These are explained in detail in §\ref{sec:basics_datasets}} for \ac{NLI} and follow the Siamese Architeture, originally introduced by \cite{bromley1994signature}, thus first encoding $p$ and $h$ using the same sentence encoder with shared weights into fixed length sentence representations and then predicting the entailment label from the combination of both representations using an additional \ac{MLP}.

\subsubsection{Sentence Encoding for Shortcut-Stacked-Encoder}
The key novelty of this approach is the way sentence representations are created using a three-layer \ac{biLSTM} with shortcut connections and row-wise max-pooling. An overview of this architecture is given in Figure \ref{fig:sentence_emcoder_shortcut}.
\begin{figure}[tph!]
\centering
	\includegraphics[totalheight=8cm]{fig/sentence_encoder_shortcut.png}
	\caption{The architecture of the sentence-encoding component within the Shortcut-Stacked-Encoder, taken from \cite{nie2017shortcut}.}
	\label{fig:sentence_emcoder_shortcut}
\end{figure}
Due to the arbitrary length of words in textual input a widely used strategy to encode variable length inputs to fixed length vectors using \ac{LSTM} \citep{hochreiter1997long} or the bidirectional variant \ac{biLSTM} \citep{graves2005framewise}. Essentially these components learn with the use of gates what information to keep and forget at a given point in time. By sequentially going through a sentence in one or two directions respectivly, are capable of exploiting word-order and take context into account.
\newline

The main difference of the Shortcut-Stacked Sentence-Encoder to typical approaches of a multi-layer \ac{biLSTM} model is that the input to the \ac{biLSTM} in a following layer is not only the output of the previous layer, but the output of \textit{all} previous layers, together with the word embeddings. In the first step, the embedding layer maps each word $\omega_i$ of the source sentence $(\omega_1, \omega_2, ..., \omega_n)$ to a $d$-dimensional word vector $w_i \in \mathbb{R}^d$. According to \cite{nie2017shortcut} we denote $x_t^i$ to be the input of the $i$th \ac{biLSTM} at timestep $t$. Naturally the input to the first layer are the word-embeddings itself, thus:
\begin{equation}
x_t^1 = w_t
\end{equation} 
In all \ac{biLSTM} with $i > 1$ the input is the concatenation of all intermediate inputs of previous layers at the timestep $t$ together with the initial word embeddings. Let $[]$ denote the vector concatenation and $h^i_t$ be the output of the $i$th \ac{biLSTM} at timestep $t$, this leads to:
\begin{equation}\label{eq:stacked_encoder_input}
x_t^i = [w_t, h_t^{i-1}, h_t^{i-2}, ... , h_t^1]
\end{equation}

Only the last \ac{biLSTM} layer is used to generate the final sentence representation. Assuming $m$ layers in total, $d_m$ to be the hidden state dimension of the last layer, that is defined as $H_m=(h_1^m, h_2^m, ..., h_n^m)$ the final sentence representation $v$ is obtained by applying row-max-ppoling over the last layer:
\begin{equation}
v = max(H^m)
\end{equation}
With each $h_i^m \in \mathbb{R}^{2d_m}$ and $H^m \in \mathbb{R}^{2d_m \times n}$ the resulting sentence vector $v \in \mathbb{R}^{2d_m}$ essentiallly captures the highest value of each dimension over all timesteps\footnote{$d_m$ is multiplicated by $2$ since the \ac{biLSTM} creates $d_m$ features for going through the sentence forwards and backwards repsectively.}.
\subsubsection{Classification}
A two-layer \ac{MLP} using ReLu as acticvation function and a final softmax-layer is used for the prediction. The input to the classifier $m$ is the concatention of the sentence representations $v_p$ and $v_h$ for $p$ and $h$ respectively together with the element-wise distance and the elementwise product, denoted as $\otimes$ of both representations:
\begin{equation}
m = [v_p, v_h, |v_p-v_h|, v_p \otimes v_h]
\end{equation}
Even thow a multi-layer \ac{MLP} theoretically would be able to learn the latter two features, \cite{mou2015natural} showed that this particular feature concatenation gives a performance gain for neural models for \ac{NLI}.
\subsubsection{Training}
For all our reimplementations using pytorch\footnote{\href{http://pytorch.org/}{http://pytorch.org/}} of the model we follow the parameters of the original paper of \cite{nie2017shortcut}. The model is trained using Adam \citep{kingma2014adam} parameter optimization, cross-entropy loss as objective function and minibatches of size 32. To avoid overfitting a dropout of 0.1 is applied on each layer of the \ac{MLP} and the accuracy is evaluated regulary on a different dataset than the train data, the development set, as it is common practive in machine learning. The final performance is estimated by evaluating the best model based on the accuracy on the development set on unseen hold-out data, the test set. 300-dimensional GloVe 840B pretrained word-embeddings \citep{pennington2014glove} are used and finetuned during training. Three additional word-vectors are added, one for unknown words, as well as one to indicate the start and one to indicate the end of a sentence. The learning rate starts with 0.0002 and is reduced by half every second iteration. We conduct our experiments with different re-implementations of this model, partly due to using fewer parameters by reducing the dimensionality of the components, partly due to changes within the original paper.
\subsubsection{Residual Encoder and Reimplementation Variants}
In a second version of the paper, \cite{nie2017shortcut} introduced the Residual Encoder, slightly adapting the way sentences are encoded. In order to create the input to the $i$th \ac{biLSTM} layer $x_t^i$ concatenation all previous outputs $(h_t^{i-1}, h_t^{i-2}, ... , h_t^1)$ together with $w_t$, naturally leads to a tremendous increase of parameters. By using residual connections, instead of concatenating all previous outputs, they are added up, thus equation (\ref{eq:stacked_encoder_input}) changes to
\begin{equation}
x_t^i = [w_t, h_t^{i-1} + h_t^{i-2} + ... + h_t^1]
\end{equation}
and reduces the parameter size.
\paragraph*{Implementation Variants}
We use the following implementations of the model. The performance comparison between the models based on SNLI\footnote{SNLI is a huge dataset for \ac{NLI} and will be explained in §\ref{sec:basics_datasets}}, is listed in Table \ref{table:reimplementation_performance} and do not differ tremendously from what \cite{gong2017natural} estimated to be the human performance on the same task.
\begin{table}[!htbp]
\begin{center}
\begin{tabular}{lccc}
\textbf{Model} & \textbf{SNLI train acc.} & \textbf{SNLI dev acc.} & \textbf{SNLI test acc.}\\
\toprule
Shortcut-Stacked Encoder\textsuperscript{$\dagger$} & 87.4\% & 85.2\% & 84.8\% \\
Shortcut-Stacked Encoder\textsuperscript{$\dagger\dagger$} & 89.4\% & 86.0\% & 85.4\% \\
Residual Encoder\textsuperscript{$\dagger$} & 91.1\% & 85.9\% & 85.8\% \\
Residual Encoder\textsuperscript{$\Diamond$} & 91.0\% & 87.0\% & 86.0\% \\
\midrule
Human Performance \citep{gong2017natural} & - & - & 87.7 \\
\bottomrule
\end{tabular}
\caption{Accuracy in percent of different implementations of the model from \cite{nie2017shortcut}, achieved on the SNLI dataset compared with human performance.}
\label{table:reimplementation_performance}
\end{center}
\end{table}
\begin{itemize}
\item We refer to Shortcut-Stacked Encoder\textsuperscript{$\dagger$} as the first re-implementation. This uses $256\times2$, $512\times2$ and $1024\times2$ dimensions for the three layers of the sentence encoding \ac{biLSTM} and $1600$ dimensions in the classifier \ac{MLP}.
\item We refer to Residual Encoder\textsuperscript{$\dagger$} when we use our own re-implementation with residual connections. The sentence-encoding \ac{biLSTM}s each have the dimensionality of $600\times2$ and the layers of the \ac{MLP} of $800$.
\item We refer to Residual Encoder\textsuperscript{$\Diamond$} when we use the final published version of \cite{nie2017shortcut} with their provided code\footnote{https://github.com/easonnie/ResEncoder}. This model has the same parameter sizes as Residual Encoder\textsuperscript{$\dagger$}.
\item We refer to the plain model name, when talking about the model structure in general.
\end{itemize}
\newpage\cleardoublepage
\section{Related Work}\label{sec:related_work}
Much work has been done to create strong models for \ac{NLI} and we show some successful strategies in Section §\ref{sec:models_snli}. Relevant datasets for \ac{NLI} are introduced in Section §\ref{sec:basics_datasets}. Before the excessive usage of neural networks, many models heavily relied on external resources, that have either been manually created in order to improve tools for \ac{NLP}, or arised in a crowd sourced manner for a different purpose, but can also be exploited. In Section §\ref{sec:ext_resources} we show an overview of some external resources that might improve the performance of neural models on \ac{NLI}. While most neural models rely solely on distributed word-representations as external information and perform quite good, prior work \citep{bos2005recognising,tatu2005semantic} depended to large degree on those resources. In Section §\ref{sec:ext_res_in_nn} we show several approaches trying to combine the power of well structured, knowledge-rich resources with the generalization power, coming from neural models with distributed word embeddings.
\subsection{External Resources}\label{sec:ext_resources}
A large variety of knowledge bases exist, containing for instance lexical relations or commonly known world knowledge, which can be helpful for improving the performance on \ac{NLI}. Research has shown that both, manually created and crowd-sourced resources, can successfully be applied in many tasks of \ac{NLP}. In this section we only show WordNet and Wikipedia, containing different information that we consider to be useful for \ac{NLI} and \ac{NLU}, as well as two resources combining multiple resources and thus providing a huge amount of readily-available knowledge.

\subsubsection{WordNet}\label{sec:wordnet}
\begin{wrapfigure}[13]{R}{0.5\textwidth}
\centering
	\includegraphics[totalheight=3.5cm]{fig/wordnet_example.png}
	\caption{Example of different synsets of the lemma ``table'' (only noun senses) within WordNet, taken from \href{http://wordnetweb.princeton.edu}{http://wordnetweb.princeton.edu}.}
	\label{fig:wordnet}
\end{wrapfigure}
WordNet \citep{miller1995wordnet} is a famous, manually created lexical resource for the English language consisting of thre lexica for four different \ac{POS}, one for verbs, one for nouns and one for adjectives and adverbs respectively \citep{Jurafsky2008May}. 
\paragraph*{Structure of WordNet} 
Mainly focusing on nouns\footnote{WordNet 3.0 contains 117,798 nouns, 11,529 verbs, 22,479 adjectives and 4,481 adverbs \citep{Jurafsky2008May}.} it differentiates between the more frequent class of \textit{common nouns} like ``table'' and \textit{instances} like ``Berlin''. All words are represented by their lemma and due to polysemy contain one or more senses, namely \textit{synsets}. Synsets are the main building blocks within the WordNet ontology, containing a sense description and examples. Figure \ref{fig:wordnet} displays 6 different senses for the lemma ``table''. It is noteworthy that the sense of table (as tabular array) greatly differs from the sense as ``furniture'' or ``tablelands'' while metaphorical senses strongly correlate with the sense of table as a furniture. Yet, they encode much more fine-grained sense-differences, that differ only slightly from each other, compared with the difference in meaning off the first synsets. While lemmata within the same synset refer to the lexical semantic relation synonymy, other lexical semantic relations like hypernomy, antonomy and holonomy (as described in Section \ref{sec:word_relations}, however more fine-grained\footnote{For example,  WordNet differentiates between \textit{hypernyms} for common nouns and \textit{instance-hypernyms} for instances, or distinguished between \textit{part- }, \textit{member-} and \textit{substance-holonyms}.} within WordNet) are defined via labelled links between synsets. Thus, WordNet holds valuable knowldge for detecting lexical inferences in natural language.

\paragraph*{Usage and Issues}
When using WordNet in applications one has to identify the correct sense out of many possible synsets for a given lemma. This may be done using proper algorithms for \ac{WSD}. Another simple and frequently used heuristic is to always choose the first snyset, which typically reflects the most common sense \citep{mccarthy2004using}. As shown in Figure \ref{fig:wordnet}, word-senses are defined with different granularities, sometimes varying only with subtle differences that are not required by most applications. Subsequently, this reduces the interpretability of path lengths of lexical relations between two synsets. For instance, identifying that ``sunflower'' is a hyponym of ``plant'' requires the traversal over five edges (\textit{sunflower $\rightarrow$ flower $\rightarrow$ angiosperm $\rightarrow$ spermatophyte $\rightarrow$ vascular plant $\rightarrow$ plant}). At the same time, identifying that a ``church'' is a ``building'' can be identified by only traversing over two edges (\textit{church $\rightarrow$ place of worship $\rightarrow$ building}) and traversing similarily over five edges leads to the synset ``whole, unit'', covering both, living things and objects. This is a known issue \citep{resnik1995using} and  strategies have been proposed to reduce the complexity of WordNet, if the specific domain is known, for instance using sense clustering \citep{prakash2007learning}.

\subsubsection{Wikipedia}\label{sec:wikipedia}
While WordNet contains manually annotated lexical relations and is easily and automatically accessable, Wikipedia\footnote{\href{https://www.wikipedia.org/}{https://www.wikipedia.org/}} is a huge multi-lingual, continuously growing  encyclopedia, maintained by many volunteers. Also mostly focusing on nouns, due to the nature of containing ecyclopedic information, it contains a large variety of factual information about named entities, that many other lexical resources lack \citep{gurevych2016linked}. Even though it has not been created for the purpose of serving as a lexical knowledge base, it still may be seen as  partially annotated resource, due to artifacts like hyperlinks. These can be interpreted similarily and even accessed using available tools in a programmatic manner \citep{zesch2008extracting}.  \cite{gurevych2016linked} describe the following information types that can be exploited to retrieve lexical information:
\begin{itemize}
\item \textbf{First paragraph:} The first paragraph of an article can be interpreted as the \textit{sense definition}, since every article covers only one aspect due to the nature of encyclopedias.
\item \textbf{Hyperlinks:} \textit{Sense examples} can be retrieved from the context, surrounding a hyperlink that links to the entity of interest, showing how the term is used.
\item \textbf{Hyperlinks:} Hyperlinks between articles can be considered as \textit{sense relations}.
\item \textbf{Translation Pages:} Due to interlinked articles in different languages, the corresponding titles usually can serve as \textit{translation equivalents}.
\end{itemize}
Wikipedia has successfully been used in many applications for \ac{NLP} and even though we do not conduct experiments within this work using Wikipedia, it clearly contains rich factual and world knowledge that can be helpful for \ac{NLI} systems. 
\subsubsection{Derived from multiple Knowledge Bases}
\ac{YAGO} \citep{suchanek2007yago} combines the high coverage of Wikipedia with the clean taxonomy of WordNet, leading to a very knowledge rich resource. \ac{YAGO} mainly targets to contain a large amount of world-knowledge with Wikipedia, as being tremendously larger than WordNet, and additionally contains relations to express facts derived from it. As opposed to \ac{YAGO}, UBY \citep{gurevych2012uby} aims for lexical semantic richness. In addition to Wikipedia and WordNet, seven other resources are combined together, providing lexical semantic knowledge in German and English. The combination is realized by using so-called \textit{sense axis}, links between two senses of different lexicons. UBY provides an easy-to-use API, making its high-coverage knowledge programatically accessable to \ac{NLP} applications. Having these knowledge-rich resources available, but for the most part de-coupled from neural approaches, still lacking this exact knowledge, stresses the benefit of combining these two worlds. 

\subsection{Datasets for NLI}\label{sec:basics_datasets}
As neural models usually require a huge amount of data for their training, they were not successfully applicable to \ac{NLI} tasks until the release of \ac{SNLI}, where they reach state-of-the-art results. Previous \ac{NLI} tasks like FraCas \citep{cooper1996using} or the PASCAL challenge \citep{dagan2006pascal} only consisted of a very limited amount of training data, such that neural models could not be used successfully. Some datasets, like \ac{SICK} \citep{marelli2014semeval} or the Denotation Graph entailment set \citep{young2014image}, increased the amount of samples at the expense of using artificially created sentences and/or automatically labeling, reducing the textual quality and adding noise. Since the focus in this work is on neural models, only the relevant datasets for this purpose are introduced.
\subsubsection{SNLI}\label{sec:snli}
With the release of \ac{SNLI} \citep{bowman2015large} researchers were able to apply neural models for the task of \ac{NLI} using distributed word. The corpus consists of 570,152 human written sentence pairs and differentiates between the three labels, described in Section §\ref{sec:basics_nli}. 
\paragraph*{Event co-reference}
A drawback of all previsously existing resources for \ac{NLI}, that is handled by \cite{bowman2015large}, is the fact that even humans may assign different labels to a sentence pair, based on their subjective interpretation of a sentence, that all can be valid. This issue can be demonstrated using the following sentence pair:
\begin{center}
\begin{tabular}{rl}
\textbf{Premise:} & Young people are demonstrating in San Francisco.
\\
\textbf{Hypothesis:} & Young people are demonstrating in New York.
\end{tabular}
\end{center}
One could clearly argue the sentence-pair should be labelled as \textit{neutral}, since there could be people demonstrating in both towns. However it is also legimate to interpret these as contradicting sentences, if one considers both sentences to be describing the same event. While both sentences may be true when describing different potential scenarios, only one of them can be true if they refer to the same. In order to reduce noise coming from these inconsistent interpretations, the labeling scheme within \ac{SNLI} must be fixed beforehand. Specifically they choose the labelling scheme to be based on event-coreference, the latter of the two explanations, as otherwise only very general statements would result in \textit{contradiction}.
\paragraph*{Generation}
In order to create \ac{SNLI}, \cite{bowman2015large} used image captions from the Flickr30k corpus \citep{young2014image} as premises and let human workers create according hypothesis for each label respectively using Amazon Mechanical Turk by asking them to write alternative captions that
\begin{itemize}
\item definetely also are a true description of the photo  \textbf{(entailment)}
\item might be a true description of the photo \textbf{(neutral)}
\item definetely are a false description of the photo \textbf{(contradiction)}
\end{itemize}
The workers only saw the image caption, not the image itself, but were encouraged to use common world knowledge, enabling the creation of inferences that require additional information of the world, that is not available in word-embeddings\footnote{For instance (taken from \ac{SNLI}) \textit{snow} is paraphrased as \textit{frozen particles of water} and requires very deep factual knowledge to be understood correctly.}. While this process simplifies the task of assuming event-coreference, the sentences within \ac{SNLI} are rather simple and short, due to the nature of image captions. 

\paragraph*{Looking into data}
As we conduct most of the experiments of this work on \ac{SNLI}, it is important to get an understanding how sentences in this dataset look like. As previously mentioned, the vast majority of sentences are rather simple and might even be phrases only, rather than proper sentences, due to omission of a verb. In addition to that, sentences might be written in proper English, but also might contain spelling or punctuation errors, be lowercased only, or describe highly unrealistic scenarios.  Table \ref{table:snli_example} shows selected sample sentence-pairs, taken from the \ac{SNLI} dataset.
\begin{center}
\begin{table}[htt]
\begin{center}
\resizebox{\textwidth}{!}{%
\begin{tabular}{lll}
\textbf{Premise} & \textbf{Hypothesis} & \textbf{Label} \\
\toprule
\multirow{3}{*}{\textbf{(1)} The large brown dog jumps into a pond.} & The dog is getting wet. & \textit{entailment} \\ & The dog is a chocolate Labrador Retriever. & \textit{neutral} \\
&A white cat is sunning itself on a windowsill. & \textit{contradiction} \\
\midrule
\multirow{3}{*}{\textbf{(2)} A woman is handing out fruit.} & A woman is passing out different types of fruits. & \textit{entailment} \\
&A woman is handing out oranges. & \textit{neutral} \\
&A fruit is handing out a woman. & \textit{contradiction} \\
\midrule
\multirow{3}{*}{\textbf{(3)} A basketball game.} & A sports game. & \textit{entailment} \\
&A basketball game between rivals. & \textit{neutral} \\
&A volleyball game. & \textit{contradiction} \\
\bottomrule
\end{tabular}}
\end{center}
\caption{Example sentence pairs, taken from \ac{SNLI}, showing typical sentences within the dataset.}
\label{table:snli_example}
\end{table}
\end{center}
The first column displays the premise, the original image caption, in the second column three hypothesis are shown, created by the workers for each label respectively. Several characteristics of the dataset and types of required knowledge to solve the task can be seen here. 
The first examples \textbf{(1)} require the model to have some factual knowledge that a ``Labrador Retriever'' is some kind of ``dog'', and ``chocolate'' is paraphrasing ``brown''. Since ``Labrador Retriever'' is a possible substitute for ``dog'' but more specific, the sample is labelled as neutral. The according entailing hypothesis requires an even deeper understanding of the world, as the system needs to know, that a ``pond'' is filled with water and anything that goes into water is ``getting wet''. The contradicting sample shows two frequently occuring characteristics. Not only has ``dog'' been replaced by ``cat'', but also the color and the activity changed. We found that in many contradicting hypothesis several contradicting words with respect to the premise exist, obviously making the task easier, as it is sufficient to only detect one several indicators. Additionally it has been shown that the creation process of the hypothesis followed some unconscious heuristics of the worker\citep{gururangan2018annotation}. Specifically the replacement of ``dog'' to ``cat'' occurs often enough, that the presence of ``cat'' in the hypothesis alone is a strong indicator for contradiction already. 
\newline

\noindent
The sentences of the second example \textbf{(2)} are based on paraphrasing, representing the same meaning, (entailment), have more specific term in the hypothesis as in the premise (neutral) and show semantic role reversal (contradiction),  which is somewhat interesting, as it requires to model to leverage word order, while a simple \ac{BoW} approach would fail here.
\newline

\noindent
In contrast to \textbf{(1)} the sentences in \textbf{(3)} only require very shallow knowledge. Here, the word ``basketball'' is substituted by it's hypernym\footnote{At least in one sense, not in the sense of \textit{being a ball}.} ``sport'', thus still covering the original meaning by being more general. The next sentence gives some plausible additional information not given in the premise, hence neutral. In the last contradicting sentence, the model has to identify that ``basketball'' and ``volleyball'' are mutually exclusive, which shows, how co-hyponyms influence the relation label. While sentences are often a bit longer than in this example, the required knowledge, as specified in \textbf{(3)}, is most present within \ac{SNLI}.

\subsubsection{MultiNLI}
\ac{SNLI} received some critizism within the research community \citep{chatzikyriakidis2017overview,williams2017broad}, mainly due to it's simplicity, cominng from the fact, that all premises are taken from a single genre only, namely image captions. Thus, \ac{SNLI} is very limited to only visual scenes, neglecting many other aspects like temporal reasoning, modality or belief. \cite{williams2017broad} introduced with \ac{MultiNLI} a new dataset, overcoming these drawbacks. 
\paragraph*{Generation of \ac{MultiNLI}}
The authors followed the same generation procedure as has been done by \cite{bowman2015large}, but instead of relying on image captions only, they took into considerations other genres from \ac{OANC}\footnote{Genres from \ac{OANC}: \textit{Government, Slate, Telephone Speech, Travel Guides, 9/11 Report, Face-to-face Speech, Letters, Nonfiction Books, Magazine}} \citep{ide2001american,ide2004american,ide2006integrating} as well as several freely available fiction work, resulting in 10 additional genres with 392,702 new sentence-pairs for training and 20,000 for development and test respectively. A major motivation for the creation of \ac{MultiNLI} was, to put more emphasis on the role of \ac{NLI} as evaluation benchmark of \ac{NLU} that \ac{SNLI} failed to provide due to its narrow coverage. Therefore, only five of the new genres are present within the train data, while the remaining five genres only occur in the test set, serving as evaluation for cross-domain transfer learning and domain adaption. The performance on this dataset is measured in two figures, \textit{matched} examples are derived from the same source as training samples, while \textit{mismatched} examples differ from those seen during the training (containing the additional genres). This motivation becomes also clear from the corresponding Shared Task \citep{nangia2017repeval}, allowing any kind of external resources (including the ones that were used to derive the premises) but only accepting sentence-encoding models\footnote{These models encode each sentence individually and are explained in Section §\ref{sec:models_snli}.} to evaluate sentence representations learning with respect to \ac{NLU}. \ac{MultiNLI} has been shown to be harder than \ac{SNLI} \citep{williams2017broad}, the best performing model of the RepEval 2017 Shared Task reaches 74.9\% matched and mismatched accuracy \citep{chen2017recurrent} using ensembles and 74.5\% matched, 73.5\% mismatched accuracy using a single model\citep{nie2017shortcut}. 
\paragraph*{Looking into data}
Table \ref{table:multinli_example} depicts a few samples of different genres\footnote{Taken from https://repeval2017.github.io/shared/}. One can see how different genres broaden the scope of language that is used to express inferences. A system needs to deal with temporal information and less visualizable terms like \textit{appreciate} or \textit{benefit}.
\begin{center}
\begin{table}[h]
\begin{center}
\resizebox{\textwidth}{!}{%
\begin{tabular}{llll}
\textbf{Premise} & \textbf{Hypothesis} & \textbf{Label}  & \textbf{Genre} \\
\toprule
\multirow{3}{*}{\parbox{6cm}{The Old One always comforted Ca'daan, except today.}} & \multirow{3}{*}{\parbox{6cm}{Ca'daan knew the Old One very well.}} & & \\
& & \textit{neutral} & Fiction \\
& & & \\
\midrule
\multirow{3}{*}{\parbox{6cm}{Your gift is appreciated by each and every student who will benefit from your generosity.}} & \multirow{3}{*}{\parbox{6cm}{Hundreds of students will benefit from your generosity.}} & &\\
&  & \textit{neutral} & Letters \\
& & & \\
\midrule
\multirow{3}{*}{\parbox{6cm}{At the other end of Pennsylvania Avenue, people began to line up for a White House tour.}} & \multirow{3}{*}{\parbox{6cm}{People formed a line at the end of Pennsylvania Avenue.}} &  & \\
& & \textit{contradiction} & 9/11 Report \\
& & & \\

\bottomrule
\end{tabular}}
\end{center}
\caption{Example sentence pairs from \ac{MultiNLI}, taken from RepEval 2017 Shared Task, showing samples of different genres.}
\label{table:multinli_example}
\end{table}
\end{center}
As the authors followed the same guidelines, as used for \ac{SNLI}, and also assume event-coreference, both datasets are highly compatible, only differing in the range of genres and thus diversity of language. In fact, \ac{MultiNLI} is even distributed in the same data format and a common practive is, to include data from \ac{SNLI} when training models for \ac{MultiNLI} \citep{nie2017shortcut,balazs2017refining,yang2017character}.

\subsubsection{SciTail}
SciTail \citep{scitail} is yet another dataset for \ac{NLI}, designed to address a different problem of previously existing datasets\footnote{Ignoring small-scale datasets with less than 1,000 samples.}. The targeted problem of previous work, including \ac{SNLI}, is, that either the premise or the hypothesis was specifically for this task created, thus neglecting the kind of naturally occuring inference problems of any endtask like \ac{QA}. It is comparably smaller, consisting of only 27,026 examples and only distinguishes between two labels, \textit{entailment} and \textit{neutral}. \textit{Entailment} is defined as in \ac{SNLI} and \ac{MultiNLI}, saying that the premise supports the hypothesis. All cases where the hypothesis is not supported by the premise however are classified \textit{neutral}.

\paragraph*{Generation of SciTail}
In order to retrieve premise and hypothesis from a resource, rather than creating one sentence for the specific purpose of \ac{NLI}, \cite{scitail} took a different approach to generate the corpus. The dataset originates from school-level multiple-choice questions for science \ac{QA} \citep{welbl2017crowdsourcing}. Those questions generally require sophisitcated reasoning capabilities in order to answer them correctly. 
\begin{enumerate}
\item \textbf{Hypothesis:} Given the short factual answer-candidates and a question, a new sentence was synthesized using the question and answer. This sentence serves as the hypothesis. For instance the question ``When waves of two different frequen-
cies interfere, \textit{what phenomenon occurs?}'' and the orrect answer ``beating'' is transformed into ``When waves of two different frequencies interfere, \textit{beating occurs}'' \citep{scitail}.
\item \textbf{Premise:} A large background corpus with relevant information from \cite{clark2016combining} was used to generate candidate knowledge sentences for each question using \ac{IR} for the premise.
\item \textbf{Label:} While hypothesis, derived from an incorrect answer, can be assumed to be not-supported by the premise, those derived from a correct answer are not necessarily supported by the sentence gained from the background corpus (the premise). Thus, samples where croud-sourced annotated, to ensure a correct labelling, only keekping those samples as entailment, that were labelled to have \textit{Complete Support}\footnote{Annotators could decide between \textit{Complete support} (labelled as entailment), \textit{Partial Support} (ignored) and \textit{Unrelated} (labelled as neutral).}.
\end{enumerate}


\paragraph*{Comparison with \ac{SNLI} and \ac{MultiNLI}}
Due to it's design, SciTail is different in nature to the two previous datasets. Neither does it contain contradicting examples, nor does assume event-coreference, as sentence-pairs in this dataset are more based on factual information. Table \ref{table:scitail_example} shows somple sentences of the SciTail dataset. Clearly all of them contain factual information, whereas in the previous shown datasets, sentences tend to be more situational. The premise can be relevant for the entailment relation, yet must not be. 
\begin{table}[!htbp]
\begin{center}
\begin{tabular}{lll}
\textbf{Premise} & \textbf{Hypothesis} & \textbf{Label} \\
\toprule
\multirow{3}{*}{\parbox{9cm}{Bones come together to form joints, most of which are in constant motion.}} & \multirow{3}{*}{\parbox{5cm}{Joints are the location where bones come together.}} & \multirow{3}{*}{\parbox{2cm}{\textit{entailment}}}\\
& &  \\
& &  \\
\multirow{4}{*}{\parbox{9cm}{Bone, Joint, and Muscle Disorders Chapter 54 Charcot's Joints Charcot's joints (neuropathic joint disease) results from nerve damage that impairs a person's ability to perceive pain coming from a joint;}} & \multirow{4}{*}{\parbox{5cm}{Joints are the location where bones come together.}} & \multirow{4}{*}{\parbox{1.5cm}{\textit{neutral}}}\\ 
& &  \\
& &  \\
& &  \\
\midrule
\multirow{3}{*}{\parbox{9cm}{The time to travel the horizontal distance (the range) is equal to twice the time to reach the peak (maximum height).}} & \multirow{3}{*}{\parbox{5cm}{Range is the maximum horizontal distance traveled by a projectile.}} & \multirow{3}{*}{\parbox{1.5cm}{\textit{entailment}}}\\
& &  \\
& &  \\
\multirow{3}{*}{\parbox{9cm}{First, finding the launch angle for maximum horizontal range in idealized projectile motion.}} & \multirow{3}{*}{\parbox{5cm}{Range is the maximum horizontal distance traveled by a projectile.}} & \multirow{3}{*}{\parbox{1.5cm}{\textit{neutral}}}\\ 
& &  \\
& &  \\
\bottomrule
\end{tabular}
\caption{Example sentence pairs from SciTail Task, different premises retrieved for two hypothesis.}
\label{table:scitail_example}
\end{center}
\end{table}
Due to its relatedness with Scientific \ac{QA}, the authors claim, that a model reaching a good performance on this dataset for \ac{NLI} will also score well on an according \ac{QA} task, as similar \ac{NLU} is needed.
\subsection{Neural Models for NLI}\label{sec:models_snli}
We follow the \ac{SNLI} leaderboard\footnote{https://nlp.stanford.edu/projects/snli/} by differentiating between \textit{sentence-encoding} and \textit{inter-sentence-attention} based models. In the following, we show an overview about relevant approaches of both areas. The Residual Encoder or Shortcut-Stacked Encoder, as introduced in Section §\ref{sec:residual_encoder_def}, belongs to the former class of models. 
\subsubsection{Sentence Encoding Models}
Sentence-encoding models follow the Siamese Architecture \citep{bromley1994signature}, meaning they encode both, sentences $p$ and $h$, individually, with parameters being tied between both sentence encoders. The inference classification is predicted by a following classifier like a \ac{MLP}. Doing so, the models put more emphasis on a meaningful sentence representation with the motivation of being more generally applicable and less focused on the specifc characteristics of the task at hand \citep{bowman2016fast}. Many different strategies are used to create meaningful sentence representations within this class of neural models. This is for instance done by exploiting syntactical information using neural Shift-Reduce-Parsers, that create a linear sequential structure from tree-structures sentence representations \citep{bowman2016fast}, or by adding and external memory with read- and write-operations, capturing the temporal and hierarchical information within natural language \citep{munkhdalai2017neural}. 

\paragraph*{Inner-attention-based models}
Following the intuition that humans only remember certain parts of a sentence after reading it, \cite{chen2017recurrent} model this human behaviour using gated intra-sentence attention, by generating the sentence representation via pooling\footnote{As done with max-pooling by \cite{nie2017shortcut}.} strategies over the outputs of the encoding \ac{biLSTM}. The outputs are reweighted using  attention gates. The idea of using inner-attention mechanisms is also used by the best performing sentence encoders for \ac{SNLI}, at the time of this writing reaching 86.3\% in accuracy \citep{shen2018reinforced,im2017distance}. \cite{shen2018reinforced} create sentence representations using the combination of hard and soft self-attention\footnote{A plain attention function calculates the alignment for an input sequence $x=[x_1, x_2, ...,x_n]$ given a query $q$. In the special case of self-attention, $q$ arises from the input sequence $x$ itself \citep{shen2018reinforced}.}. While hard-attention forces the model to only focus on relevant elements of the input sequence, disregarding all other elements, it is not fully differentialbly and thus inefficient to train. Soft-attention methods on the other hand, are fully differentiable and weight each element of the input sequence accoridng to their relevance. However, by also giving positive, non-zero weights to irrelevant elements, it diminishes the emphasis on truely important ones. By first applying hard-attention to retrieve a subset of context-aware elements, that is afterwards processed using soft-attention, \cite{shen2018reinforced} leverage the mentioned advantages both techniques. Inner attention is also used by \cite{im2017distance}, however their model additionally uses directional masks, that prevent the network from considering following or predeeding words in the attiontion proces respectively. Furthermore, they use distance masks, that reduce the attention weights, if words are further away to each other. They show that their model outperforms others, especially with longer input sentences, as the result of considering word distance and positional information.

\subsubsection{Inter-sentence-attention-based models}\label{sec:rel_work_sentence_encoding_models}
\cite{rocktaschel2015reasoning} shows, that models perform significantly better, when looking at both sentences simultaneously in the senteence-encoding step. This is motivated by the way, humans would solve the task of \ac{NLI}, by first reading the premise, and creating the understanding of the hypothesis based on the previously read sentence. Since this seems to be superior in \ac{SNLI}, many works follow this approach reaching state-of-the-art results. Also in this class of methods memory networks, accessable via attention, were applied \citep{cheng2016long}.

\paragraph*{Inter-sentence-attention-based models used within this work}
\cite{parikh2016decomposable} provide a simple network structure, called \textit{Decomposable Attention}, using the assumption that only parts of a sentence are needed for the entailment relationship. They do so by fragmenting the input sentences into subphrases and align the fragments of both sentences with each other using attention. Even though they represent sentences in a \ac{BoW} manner, they reach a remarkable performance. After comparing the aligned phrase-pairs, the final sentence representation is retrieved by a simple summation over the comparison-vectors from the previous step. Thus, by using this rather basic aggregation method rather than relying on any \ac{LSTM}-based method, they reduce computational complexity tremendeously. \ac{ESIM} \citep{chen2017enhanced} is another simple yet strong model, essentially consisting of three different steps. First, words are encoded using \ac{biLSTM}s such that they represent the context as well as the word itself. Next, similarily to \cite{parikh2016decomposable}, they calculate the local inferance between elements in both sentences, by reweighting the sentence representations, based on the normalized attention weights. They enhance this information, using the feature concatenation of \cite{mou2015natural}, as done in the Shortcut-Stacked Encoder, however in this approach word order information is preserved by the network, in contrast to Decomposable Attention. The final sentence representation is created using pooling\footnote{As opposed to summation in Decomposable Attention. \cite{chen2017enhanced} evaluate in their experiments, that pooling leads to superior results than summation, due to being less sensitive to the sentence length.} on the ouput of a \ac{biLSTM}, composing the local inference information from the previous step. The composed vector is finally fed into a \ac{MLP} classifier for the prediction. \cite{chen2017enhanced} report their results using an ensemble of two inplementations with the same base architeccture. One, as described here, relies on a \ac{biLSTM}, the other focuses more on syntactic features by encoding sentences with a TreeLSTM. While Decomposable Attention and ESIM achieve competitive results on \ac{SNLI} we conduct experiments using both models in Section §\ref{sec:additional_snli_set}, showing that these results are rather a matter of memorization than generalization.

\paragraph*{Attempt to incorporate WordNet}
Very recently, \cite{chen2017natural} introduced with \ac{KIM} a neural model, incorporating information from WordNet. This is, at the time of this writing,  the single best performing model on \ac{SNLI}. In their approach, they map WordNet relations, as defined in Section §\ref{sec:word_relations}, to a real number $r \in \mathbb{R}$ with $0 \leq r \leq 1$, quantifying the relations between word within $p$ and $h$ based on the path length of each relation within WordNet, and represent each word-pair with this additional feature vector. However evven by enrichening the representations with WordNet information, they only outperform models without external information by a small margin, ranging from 0.1 to 0.6 points in accuracy. \cite{chen2017natural} show that adding WordNet is helpful if less train data is available, however only show limited evidence, that the model leverages from WordNet fused relations for the overall improvement in accuracy\footnote{This is true for the first published version \citep{chen2017natural}. Subsequentially to work presented within this thesis in Section §\ref{sec:additional_snli_set}, they show indeed that the additional information from WordNet is a key factor within \ac{KIM} in their updated version \citep{chen-EtAl:2017b:natural}.}. In this paper we show that performance on \ac{SNLI} is not sufficient evidence for the capability of dealing with simple lexical inferences as inferred from WordNet, which suggests that further investigations should be conducted in this direction.

\paragraph*{Benchmark}
To this date, the best single sentence-encoding models on \ac{SNLI} reach 88.6\% \citep{chen2017natural} ensembles reach up to 89.3\% \citep{tay2017compare,peters2018deep,ghaeini2018dr} giving an advantage of 2.3\% or 3.0\% respectively over the best sentence-encoding model \citep{im2017distance}. Considering that the human performance on \ac{SNLI} only is estimated to be 87.7\% \citep{gong2017natural} indicates, that research started to slightly overfit on the dataset already. 

\subsection{Integration of external Resources into Neural Networks}\label{sec:ext_res_in_nn}
There have been several approaches to integrate knowledge of different kind (as described in Section §\ref{sec:basics_datasets}) into neural networks. \cite{hu2016deep} infer external knowledge, represented in logical form, using a student-teacher setup. In this setting, the teacher, being a neural network, is constrained by the rules acquired from an external resource, the student, also being a neural network, considers both labels: the true labels and the constrained predictions of the teacher. By simultaneous training both networks influenced by each others predictions, the logical rules are integrated within the networks parameters, weighted by their learned relevance (soft rules rrather than scrict hard rules). Most attempts to incoprorate external information however, do so by enhancing word representations.
\subsubsection{Improving word-embeddings}\label{sec:embeddings_improvements_relwork}
A very intuitive way to integrate external resources is, to enrich word-embeddings with additional information. As most neural models depend on vector representations for words anyway, any imrovement of word-representations can be adapted with very limited effort to most models.

\paragraph*{Joint learning of distributional embeddings with external information}
\cite{xu2014rc} differentiate between \textit{categorical} (attributes of words like the ``gender'') and \textit{relational} (relations between words, like ``child-of'', ``is-a'', e.t.c.) knowledge and train the word-embeddings from scratch, using three objective functions simultaneously. They use skip-gram\footnote{Skip-gram essentially optimizes the prediction of the probability of context-words appearing in the (close) context of a target word \citep{goldberg2014word2vec}.} to encode distributional properties. At the same time, they minimize the distance between words, that share the same category, thus clustering words by their categorical similarities. Third, they represent a relation as a vect or $r$,  and optimize word vectors, such that for a word $w_1$, connected to another word $w_2$ via relation $r$ the equation $w_1 + r \simeq w_2$ holds. \cite{liu2015learning} construct enriched embeddings by defining it as a constrained optimization problem. Specifically, they create constraints by ranking word similarities such that for instance synonyms should be more similar than antonyms or hyponyms should be more similar to close hypernyms than to distant hypernyms. Finally they include those contraints into the training process with skip-gram. 
\paragraph*{Post-processing existing representations}
 \cite{faruqui2015retrofitting} propose a method called \textit{Retrofitting}, a post-processing method than can be applied on any pre-trained word-representations. They reduce the eucledian distance between words, that are connected with a lexical semantic relation within a resource, while also keeping the representations close to the original neighbouring word-representations. Attract-Repel \citep{mrkvsic2017semantic} is another retrofitting method, essentially pulling synonyms closer to each other while pushing antonyms further apart in vector space, while trying to keep the original distributional information. Similarily \cite{vulic2017specialising} build on attract-repel, adding hypernym relations for the context of lexical entailment, by using an asymmetric distance measure between hypernym-hyponym pairs.
 
\paragraph*{Effectiveness of improved representations}
The demand of integrating lexical resources such as WordNet has mainly been targeted by enrichening word-representations, with previously mentioned approaches being just a small selection. The improvement over standard distibutional word-embeddings of most of these approaches however, is either demonstrated by visualizing word-relation vectors, that may not even be expoited by end-to-end neural networks \citep{levy2015improving}, or based on evaluations on very low-level tasks like Word-Similarity, Syntactic Relations or Analogical Reasoning, or, by solely provide intrinsic evaluations. Neural networks for higher level tasks like \ac{NLI} however reach state-of-the-art performances, still relying on standard pre-trained distributional word-representations like GloVe, even though alternatives exist. Premimitary experiments\footnote{These experiments have been conducted by Vered Shwartz in prior work and are not part of this work.} of using enriched embeddings for \ac{SNLI} have shown no success. We evaluate the possibility of adding enriched embeddings, following the successful idea of \cite{ruckle2018concatenated}, that different embeddings encode complementary information, by concatenating different word-representations. However we focus our experiments on the integration of knowledge on a more progressed step of the network, the sentence representation, due to limited reported success on end tasks using richer embeddings, though many of those embeddings exist.
%\citep{iacobacci2015sensembed} sense embeddings + drawback polysemy
%\citep{rothe2015autoextend} autoexten +WordNet (i think)
%\citep{toutanova2015representing} jointly and \citep{zhong2015aligning}
\newpage\cleardoublepage
\section{Understanding Shortcut-Stacked-Encoder}\label{sec:understanding}
In this section we give analyse the sentence representations of Shortcut-Stacked Encoder\textsuperscript{$\dagger$} by visualizing how they encode (Section §\ref{sec:insights_sent_repr}) and leverage (Section §\ref{sec:insights_sent_alignment}) information from natural language text, coming from \ac{SNLI}. Additionally we show experiments, underlining the presented insights.
\subsection{Motivation}
The major downside of neural networks is the lack of interpretability \citep{goldberg2017Apr}, thus their capabilities on a lower level can only be estimated by finding meaningful evidence for their failures or sucesses on the task at hand. While analysing errors may lead to conclusions \textit{what} does not work, \textit{why} it does not work is in many cases left to intuition. Other machine-leanring classes like probabilistic or symbolic techniques do not suffer from this problem, leading to an increasing interest in visualization techniques for neural networks. Most visualitations of sentence-representations to date focus on attention-based approaches showing how words are aligned to each other such as by \cite{shen2018reinforced} or \cite{im2017distance}. To the best of our knowledge, no insights have been gained to understand the final sentence representation in vector space. In this section we demonstrate how this representation, arising from max-pooling, can be interpreted, using the Shortcut-Stacked Encoder\textsuperscript{$\dagger$} as the model to analyze. Intuitively, understanding how the Shortcut-Stacked Encoder\textsuperscript{$\dagger$} encodes information can be helpful for the task at hand of improving it using external resources. 
\newline

\noindent
While we did not manage to leverage the insights gained in this chapter to increase the performance, it might be helpful for future work.
NN nicht gut interpretierbar

\subsection{Insights on the sentence representation}\label{sec:insights_sent_repr}
In this section we show how we analyse the informaation that is present within the sentence representions, what kind of information is encoded and demonstrate, that the sentence representation can manually be adjusted in a meaningful way.
\subsubsection{Approach}
We use Shortcut-Stacked Encoder\textsuperscript{$\dagger$}, trained on \ac{SNLI}, for our analyses. This model creates for input each sentence $x$, cosnisting of words, represented as $x_i$, a sentence representation $r \in \mathbb{R}^{2048}$ with $r_j$ being the $j$th dimension of $r$. Arising from $x$, $r$ captures the relevant information for the task at hand and is used in many neural networks without a deeper understanding what each $r_j$ actually encodes. We shed light into the dimensionewise meaning of the sentence represnetation by identifying which word is responsible for the actual value of $r_j$. 
\paragraph*{Method}\label{sec:understanding1_method}
For simplicity, We explain our applied method and the reason why we use Shortcut-Stacked Encoder using a more general neural architecture of \ac{LSTM}s, a simple uni-directional \ac{RNN}.
Figure \ref{fig:rnn} (left) shows the recursive workflow of such a \ac{RNN}, following the notations of \cite{goldberg2017Apr}.
\begin{figure}[tph!]
\centering
	\includegraphics[totalheight=5.5cm]{fig/rnn.png}
	\caption{General architecture of a \ac{RNN} (left). Example sentence in an unrolled \ac{RNN} (right).}
	\label{fig:rnn}
\end{figure}
Maintaining an internal state $s \in \mathbb{R}^m$, for $m$-dimensional representations, the network recursively iterates over the input sequence $x$, aggregating in each timestep the previous state $s_{i-1} \in \mathbb{R}^m$ with the current input $x_i$ using the function $F$. This state is then used for the next iteration and output via a mapping function as $y_i \in \mathbb{R}^m$. Multiple implementation variants exist of $F$ and what is shared across iterations. \ac{LSTM}s for instance use several neural gates to learn what information should be used, output or forgotten. This procedure ca be seen with an example sentence yb unrolling the network in Figure \ref{fig:rnn} (right). In typial setups a neural network may either choose to use $s_t$ or $y_t$ for a sequence length of $t$ as the final sentence representation \citep{goldberg2017Apr}, since the network iterated over the full input sequence and contains the relevent information, if optimized for it. Even though the architecture of different versions of \ac{RNN} may be well understood and has a logical meaning, the actual procedure of deriving concrete representations within a trained model is hard to understand. We leverage the fact that the Shortcurt Stacked Encoder uses max-pooling over all $y_i$ to gather the sentence representation rather than using $y_t$ or $s_t$ by identifying what $y_t$ has the highest value within a given dimension and mapping this dimension to the word $x_t$ of the input sentence. As an example consider the sentence in Figure \ref{fig:rnn} (right). For each timestep $t$ a new vector $y_t$ is produced. As done by \cite{nie2017shortcut} we concatenate all $y_t$ to a matrix $\mathbb{R}^{m \times t}$, with $m$ being the representation size and each vector $y_t$ being the $t$th row within $M$. Assuming a dimensionality of $m = 3$, an examplatory matrix $M$ for the given sentence ``A child is swimming .'' is displayed in Figure \ref{fig:example_process_understanding}. 
\begin{figure}[tph!]
\centering
	\includegraphics[totalheight=2cm]{fig/example_process_understanding.png}
	\caption{Visualized example of extracting interpretable information of the max-pooled seentence representations with a dimensionality of 3.}
	\label{fig:example_process_understanding}
\end{figure}
Additiobnally to creating the sentence representation $r$ by applying wor-wise max-pooling on $M$, we collect the vector $a$, containing the column indizes, that are responsible for the values within $r$. These can directly be mapped to the word of the source sentence and thus be interpreted by humans. It should be notted that due to the nature of the multi layer \ac{biLSTM} each $y_t$ does not only contain the word at $x_t$ but its context. While this somehow may lead to less accurate mapping, we found that the chosen method is sufficient to gain some meaningful insights on sentence encoding.

\paragraph*{Analysed data}\label{sec:understanding1_analysed_data}
To reduce noise and aming for sentences that Shortcut-Stacked Encoder\textsuperscript{$\dagger$} seems to have a proper understanging about, we sample 1000 sentence representations from the \ac{SNLI} train data in the following strategy. We group all sentence pairs ($p$, $h$) sharing the same premise and only keep groups if all samples belonging to the same group are classified correctly. Thus, we reduce the amount of sentences that are definetly misunderstood by the model, that would be harder to interpret. For now we are not interested in the actual relation between $p$ and $h$ and therefore create a pool of the remaining sentences, by treating $p$ and $h$ equally and splitting their connections apart. After removing duplicate sentences, the most frequent sentence length for the remaining representations is 8. To reduce noise that may arise from different sentence lengths, we only consider sentences of a length of 8 and randomly sample 1000 sentence representations. All experiments in this chapter are based on the same instances, unless otherwise stated.
\newline

\noindent
In addition to the representation values each sample contains the following information:
\begin{itemize}
\item \textbf{Token:} The tokens that triggered the maximum value for the representation.
\item \textbf{Token position:} Positional information about the responsible tokens within the sentence.
\item \textbf{Lemma:} The lemmata of the responsible tokens.
\item \textbf{\ac{POS}:} The \ac{POS} tags of the responsible tokens.
\item \textbf{Dependency Parse:} The tags of the responsible tokens within the dependency parse tree.
\end{itemize}
Lemmatizing, \ac{POS}-Tagging and dependency parsing were conducted using spaCy\footnote{\href{https://spacy.io/}{https://spacy.io/}}.

\subsubsection{Detection of relevant dimensions}
As commonly done when analysing data we start by showing a rough look into the sentence representations at hand. Typically, the \ac{SD} within a dimension correlates with the with the relevance for decision making. Naturally, a dimension that does not change its value and thus being close to a constant is not informative, while a value with a high \ac{SD} can be considered informative \citep{Bishop2007}. We calculate \ac{SD} over all dimensions, depicted as a histogram in Figure \ref{fig:sd}.
\begin{figure}[tph!]
\centering
	\includegraphics[totalheight=8cm]{fig/sd.png}
	\caption{The standard deviation within a dimension of sentence representations (x-axis) by the amount of dimensions with the given standard deviation.}
	\label{fig:sd}
\end{figure}
We plot the standard deviations in a discrete space using a bin size 0.05. For each if the 2048 dimensions we calculate its \ac{SD} to assign them to the correct bin. The amount of dimensions with the given \ac{SD} is shown on the y-axis, note that the upper part of the plot is truncated for the sake of compactness. As can be seen, only a very tiny fraction of the dimension shows a large variation, the vast majority contains more or less the same value, regardless of the sentence. This obviously  does not mean, they contain no information at all, as they may only be used to encode information that is rarly present within the data, however itserves as a reliable source, what dimensions are relevant to the model.

\paragraph*{A naive approach to identify dimensional encoding}
An intuitive approach to identify, what is encoded within the sentence representation, is to find common similarities between the words across all sentences, that are responsible for the according dimmension. Especially the task of \ac{NLI} we assume \textit{semantic}, \textit{syntactic} or \textit{positional} information to be required. Those can all be inferred using the features we extraced in Section §\ref{sec:understanding1_analysed_data}. Similarities between words heavily depend on the context they appear in \citep{dagan2000contextual}. For instance one could consider a car and an identical recunstruction in original size of the same car as similar, whereas a horse is very distinct. Adding additional information that one needs to reach a destination in short time, he or she is more likely to consider the horse similar to the car, desicing between these two option. This essentially comes to a major problem when investigating semantic encoding without prior knowledge of what attributes may be considered relevant. We therefore investigate the sentence representation using excessive manual analyses in a top down manner, by first searching for patterns across all dimensions. In Section §\ref{sec:understanding2} we will look into some dimensions in detail.
\paragraph*{A tool for sentence representation visualization}
In order to evaluate many patterns with minimal time effort, we create a visualitazion tool, capable of dynamically generating any labelling scheme for responsible words based on the features descried previously. A sample visualitation is shown in Figure \ref{fig:find_position_1}.
\begin{figure}[tph!]
\centering
	\includegraphics[totalheight=4cm]{fig/finpone.png}
	\caption{An extraction of a grid-plot, showing dimensions with the position within the sentence of the word, responsible for the dimensional value.}
	\label{fig:find_position_1}
\end{figure}
This grid-plot visualizes for each row the responsible words for one dimension, listed on the right side as (<rank in terms of \ac{SD}\footnote{All dimensions are ranked by their \ac{SD}, giving an intuition of the expressiveness of the dimension.}>, <dimension index>), colored based on the attributes of interest. In this particular case words are colored by their position within the sentence.  Each column refers to the same sentence along different dimensions. As a trade-off between explanatory power and clarity we always plot 300 sentences on 300 dimensions, which are eigther oredered by \ac{SD} or already pre-sorted by the frequency\footnote{Even though we only use 300 sentences and dimensions for plotting, calculatiions are based on all the selected data.} of a label of interest. In this particular case, dimensions are ordered by their frequency of words on the üposition 1, meaning the upmost dimension received its values from the second word (index $1$) more than any other dimension. Looking for patterns across many sentences, we focus on horizontal lines with the same coloring. Vertical lines indicate different differences across sentences w.r.t. the attribute of interest.

Filtern + Select by SD, Search


create tool
label data and sort by label freqency + SD + dimension positions

responsible word i nitcht genug, value is immer wichtig
\subsubsection{Dimension-wise Analysis}\label{sec:understanding2}
\paragraph{Positional information}
\paragraph{Semantic information}
\paragraph{Syntactic information}
\paragraph{Evaluation of the impact of female and male dimensions}
\subsubsection{Conclusion}
weg??
\subsection{Insights on the sentence alignment}\label{sec:insights_sent_alignment}
\subsubsection{Approach}
\subsubsection{Entailment analysis}
\subsubsection{Neutral and contradiction analysis}
\subsubsection{Experiments}
\subsubsection{Conclusion}
- eher experimental, need different models w maxpooling, mehr daten, mehr experiments, ...
\subsection{Errors of the base model}\newpage\cleardoublepage
\section{Additional SNLI test-set}\label{sec:additional_snli_set}
\subsection{Motivation}
\subsection{Dataset}
\subsubsection{Creation}
Upward/Downwar monoton \citep{maccartney2007natural} \citep{cooper1996role} https://www.illc.uva.nl/Research/Publications/Reports/PP-2008-05.text.pdf
\subsubsection{Validation}
\subsubsection{Final dataset}
\subsection{Other models}
\subsubsection{ESIM}
\subsubsection{Decomposable Attention}
\subsection{Evaluation}
\subsection{Analysis}
\newpage\cleardoublepage
\section{Approaches to incorporate WordNet information}\label{sec:approaches_ext_res}
Having the dataset of the previous section, we next try to improve our latest re-implementation Residual-Stacked Encoder\textsuperscript{$\dagger$} using WordNet. 
\subsection{Methods}
Unlike \ac{KIM}, which has shown an intuitive and successful strategy of incorporating WordNet, the Residual-Stacked Encoder does not use inter-sentence attention. Without changing this, we therefore cannot likewise align words of $p$ with words of $h$ to identify their WordNet relation. We intend to leave the model with the plain sentence-encoding architecture, targeting the incorporation of external resources for general sentence representations, encoding each sentence individually \citep{nangia2017repeval}. Naturally, this poses a new difficulty, since the relations of WordNet are defined between two senses. Subsequentially, we apply other strategies, than directly encoding the relation of two words (or senses), explained below.
\subsubsection{Drawbacks of using insights of max-pooled sentence representations}
In Section §\ref{sec:understanding} we gained valid insights on the sentence-representations, and showed that these sucessfully can be used to change the meaning of sentence representations. Following these conclusions, a possible strategy is, to train the model in a way, that antonyms or co-hyponyms result in distinct high dimensions, synonyms in the same high dimensions and hypernyms in a subset of high dimensions compared to hyponyms. Knowing reasonable values for each values within the dimensions, this could be broken down to a simple regression problem. Since we did not find an elegant way to naturally include this into the loss function, the only remaining strategies highly reassemble traditional feature-engineering, as the $\xi$ would need to be determined beforehand. Since the automatic feature selection is one of the key strengths of neural models \citep{bengio2013representation}, those strategies would rather be similar to a step backward than forward. Instead we identify to potential strategies, that are simpler to implement and would result in a broader applicability, not being tied to max-pooled sentence representations.

\subsubsection{Fuse WordNet information within the embedding-layer}
Additional information within the word-representations has the advantage, of being generally applicable. Following \cite{ruckle2018concatenated} we do not use exclusively retrained or adjusted word-embeddings. Instead, for each word $w$ we look up the word-vector within the original distributed GloVe embeddings and concatenate it with the corresponding word-vector of the same $w$ from the additional word-embeddings. If no vector for $w$ is present within those, we concatenate a zero-valued vector of the same dimensionality. Thus, we do not limit the original information of distributed word-embeddings and the model may still rely on the same features. Even though some of the additional features might be redundant w.r.t. the orignal GloVe embeddings, some contain additional information, that the network can use to differentiate between words, that are highly similar in GloVe. Additional to doing this experiment with the mono-lingual attract-repel vectors, provided by \cite{ruckle2018concatenated}, we use two different word-vector sources.

\paragraph*{Overfitting WordNet}
We apply a simple method to create addtitional word vectors $v$ that are similar for the words $w_1$ and $w_2$, if they are synonyms, and distinct if $w_1$ and $w_2$ are antonyms or co-hyponyms. For this we extract samples ($w_1$, $w_2$, \texttt{relation}), whereas $w_1$ and $w_2$ are lemmata, that are linked via \texttt{relation} within WordNet, represented by their GloVe embeddings. Using a two layer \ac{MLP}, we map each word-vector $w \in \mathbb{R}^{300}$ to $v \in \mathbb{R}^{20}$. In our last layer we apply $\tanh$ as non-linearity, to squeeze all values $v^i$ within $v$, with $v^i$ being the $i$th value within $v$, are in an appropriate range:  $ \forall i: [i \in \{x \in \mathbb{N} | x < 20\} \Rightarrow v^i \in \{x \in \mathbb{R} | -1 < x < 1\}]$. Let $w \in \mathbb{R}^{300 \times 1}$ be the GloVe word-embedding, $W_1 \in \mathbb{R}^{100 \times 300}$ and $b_1 \in \mathbb{R}^{100 \times 1}$ the weight matrix and bias of the first layer, and $W_2 \in \mathbb{R}^{20 \times 100}$ and $b_2 \in \mathbb{R}^{20 \times 1}$ of the second layer respectively. The new word-vector $v$ is calculated as:
\begin{equation}
v = \tanh(W_2 \text{ reLU}(W_1w + b_1) + b_2)
\end{equation}
We optimize the representations $v_1$ and $v_2$, coming from ($w_1$, $w_2$, \texttt{relation}) using \ac{MSE} with the Eucledian Distance, which should be high, if the relation indicates, $w_1$ and $w_2$ are mutually exclusive, and low, if both are synonyms. We define $\theta=0$ for synonyms, and $\theta =10$, for antonyms and co-hyponyms respectively (we bound the difference to $\frac{|v|}{2}$, with $|v|$ being the amount of dimensions of the new word-vectors, as it creates sufficiently distinct vectors) and calculate the loss as:
\begin{equation}\label{eq:loss_embd}
\text{loss} = \frac{1}{2}\Bigg( \sqrt{\sum^{20}_{i=1}(v^i_1 - v^i_2)^2} - \theta\Bigg)^2
\end{equation}
We overfit on the lexical relations extracted from WordNet, intending to memorize whether two words are compatible or not. This optimization process also updates the GloVe embeddings of $w$ during training. In order to create word-vectors that specifically focus on either antonymy or co-hyponomy, we train embeddings for each of those relations individually, both times together with synonyms in order to have a counterpart. Embeddings differentiating between synonyms and antonyms are referred to as \textit{Trained-Syn-Ant}, differentiation between synonyms and co-hyponyms as \textit{Trained-Syn-Cohyp}. We refer to the concatenation of both embeddings as \textit{Trained-Syn-Cohyp-Ant}. Since the Eucledian Distance is a symmetric measure we cannot include the hypernym-hyponym relation into those embeddings. We thus train other embeddings, differentiation between all relevant lexical semantic relations\footnote{Synonymy, Antonomy, Hypernyomy, Co-hyponomy} in a slightly adapted manner, and with 50 instead of 20 dimensions. In order to enable the network to deal with asymmetric relations, we apply a softmax layer and train the network with cross-entropy loss, predicting the actual relation, holding between $w_1$ as the first input and $w_2$ as the second input. We refer to those embeddings as \textit{Trained-All}.
\paragraph*{Adding categorical information}
Alternatively, especially targeting the detection of co-hyponyms, instead of concatenating different embeddings, we concatenate each word $w$ with the distributed GloVe vector of the hypernyms of each $w$. Specifically, for all hypernyms up to a given edge length, we take the average of all word-representations (if they exist in GloVe) of lemmata within those synsets. The motivation is, that the network is able to identify, that two words share the same hypernym. We refer to those embeddings as \textit{Hypernyms-<amount-of-hypernym-edges>}.
\subsubsection{Fuse WordNet information within the sentence-representations}\label{sec:mt_learning_intro}
It is very known, that neural networks do well in learning relevant features \citep{bengio2013representation}, however, as seen in Section §\ref{sec:additional_snli_set} and shown by \cite{gururangan2018annotation}, those features do not necessarily correspond with \ac{NLU}, but are heavily biased by dataset-specific patterns. These are not reduced, if we add additional information to the embeddings. Thus we still rely on the full model to pick up on good features, yielding to correct decisions for the \textit{right} reasons. \cite{gulccehre2016knowledge} show on a very different task, of detecting pentomino shapes, that deep neural networks may not even find the most useful features and can heavily leverage from human guidance when creating intermediate representations. In their very simple toy-scenario, this could be done by manually creating intermediate target representations, which is not easily possible for sentence-representations in \ac{NLP}. We thus continue training the neural network in an end-to-end manner. In order to guide the network in learning more useful sentence-representations, we create a second task, namely the helper-task, sharing some basic components with the maintask (which still predicts the label for \ac{NLI}). Both tasks rely on the same sentence representation, that will therefore encode relevant features for both tasks, whereas one can leverage from the features from the other. This is commonly known as multitask-learning and has shown to be successful to improve the generalization of shared representations \citep{nangia2017repeval}.
\paragraph*{Multitask architecture}
The multitask setup is visualized in Figure \ref{fig:mt_architecture}.
\begin{figure}[tph!]
\centering
	\includegraphics[totalheight=5.5cm]{fig/mt_architecture.png}
	\caption{Architecture of the Residual-Stacked Encoder with multitask learning for the sentence-representations.}
	\label{fig:mt_architecture}
\end{figure}
The left side shows the standard architecture of the Residual-Stacked Encoder, as defined in Section §\ref{sec:residual_encoder_def}. Both sentences $p$ and $h$ are encoded using the same sentence encoder. The resulting sentence-representations are concatenated with the additional features and classified by the final \ac{MLP} into on the three labels entailment, neutral or contradiction. The additonal \ac{MLP} on the right side is used for the helper task. We create the helper-task with the intention to force the model to encode differences between two words $w_1$ and $w_2$, if one is the antonym or co-hyponym of the other.  Likewise, if $w_1$ and $w_2$ are synonyms or $w_2$ is the hypernym of $w_1$ and thus entailed by it, we want this information to be encoded as well. We define our helper task as a binary classification problem, whether a word (or its meaning) is encoded (or entailed) within the sentence representation or not. For this, we consider both sentences $p$ and $h$ from a \ac{BoW} perspective. Let $S=\{w^0, w^1 , \ldots, w^{n-1}, w^n\}$ be the set of all $n$ distinct words $w$ within a sentence. We apply the same task for $p$ and $h$. Since we do not consider them simultaneously but individually, we define the helper-task using the general $S$ respectively for both. For each $w^i \in S$, we identify words from WordNet, being linked to $w^i$ with one of the previously mentioned relations. Let $A$ be the set of words, whos meaning must be entailed by the sentence representation, thus $A$ contains all hypernyms and synonyms of all $w^i \in S$. Similarily, let $B$ be the set of words, who's meaning is \textit{not} entailed by the sentence, thus antonyms and co-hyponyms of all $w^i \in S$. Additionally, all $w^i$ that have related words via lexical semantic relations are also added to $A$. A sentence like ``People are watching a soccer game between Brazil and Mexico.'' may still cause conflicts, as $A$ contains ``Brazil'' and ``Mexico'', however they may also be present within $B$, being mutual co-hyponyms. To avoid conflicts, if several co-hyponyms are present within the same sentence, we ensure that $A$ and $B$ are not overlapping, by setting $B = B \setminus (S \cup A)$. The final helper-task takes a sentence-representation $r$ and a word embedding $e$ as input, and must classify, whether $e$ belongs to $A$ or $B$, meaning whether $e$ is present (in the sense of entailment) within $r$ or not. Since the same embeddings are used and fine-tuned, this may also be seen as a postprocessing step for word vectors like in Attract-Repel \citep{vulic2017specialising}, but additionally ensuring, those differences are propagated into the sentence-representation. 

\paragraph*{Training}
The main-task and the helper-task are simultaneously trained. Thus, the combined loss, denoted as $loss_\text{combined}$, aggregates the loss for the main-task, denoted as $loss_\text{main}$, and for the helper task, denoted as $loss_\text{helper}$:
\begin{equation}
\label{eq:multitask_aggregate}
\text{loss}_\text{combined} = \alpha \text{ loss}_\text{main} + (1 - \alpha)\text{ loss}_\text{helper}
\end{equation}
Here, $\alpha \in \{x \in \mathbb{R} | 0 \leq x \leq 1\}$ regulates the impact of the main-task, with a high value ($\alpha = 1$) only considering the main-task and a low value ($\alpha = 0$) only the helper-task. While $loss_\text{main}$ remains the original mean cross-entropy, $loss_\text{helper}$ is also based on mean cross-entropy, yet is down-weighted for the following reason. Let $A_p$, $B_p$ and $A_h$, $B_h$ be $A$ and $B$ according to the definitions above for $p$ and $h$ respectively and $|A|$ denote the amount of samples within a set $A$. One can safely assume, that the amount of samples for the helper-task is tremendously higher than for the main task, since one single sample ($p$,$h$) in this task yields $|A_p|+|B_p|+|A_h|+|B_h| \gg 1$ samples in the helper-task. Let $b$ be the batch-size and $p^i$ and $h^i$ denote the $i$th $p$ or $h$ within a minibatch. We calculate $n$ to be the total amount of samples for the helper-task within a given batch:
\begin{equation}
n = \sum^b_{i=1}\Big(|A_{p^i}|+|B_{p^i}|+|A_{h^i}|+|B_{h^i}|\Big)
\end{equation}
Let $loss_\text{s}$ be the loss, calculated with mean cross-entropy, for all samples coming from $A_s$, $B_s$, with $s$ being any sentence $p$ or $h$. The re-weighted $loss_\text{helper}$ over all ($p$, $h$) within a minibatch is calculated as:
\begin{equation}
\text{loss}_\text{helper} = \sum^b_{i=1}\bigg( \frac{|A_{p^i}| + |B_{p^i}|}{n} \text{loss}_{\text{p}^i} + \frac{|A_{h^i}| + |B_{h^i}|}{n} \text{loss}_{\text{h}^i} \bigg)
\end{equation}

\paragraph*{Multitask variations}
We evaluate several implementations with small differences or changed hyperparameters, following the presented architecture. Those are described below, mostly differing in their impact on the sentence-representation. 
\begin{itemize}
\item \textbf{Size and amount of layers:} We evaluate different sizes of the helper-task \ac{MLP}. Naturally, the simpler the helper-network (fewer layers or dimensions), the more information must be encoded within the representation. This should be preferrable, since finally we aim for the main-task to leverage from the same, hopefully meaningful, features, and we do not have further use of the \ac{MLP} of the helper-task.
\item  \textbf{Dropout:} Similarily, by using dropout (0.1) in the helper-task, we motivate the creation of redundant features within the sentence-representation.
\item \textbf{Re-sample less frequent label:} We observe that usually $|A_s| < |B_s|$, resulting especially from the large amount of co-hyponyms. In order to prevent the helper task to take the label distribution into account rather than creating relevant features, we re-sample the less frequent class of $A_s$ or $B_s$, such that $|A_s| = |B_s|$.
\item \textbf{Reweighting tasks:} By either statically adapting the value of $\alpha$ beforehand, or dynamically updating it during training, we change the impact of each task. Specifically, in the \textit{finetune}\footnote{$\alpha$ for 10 iterations: $\alpha = [0.5, 0.5, 0.5, 0.5, 0.5, 0.75, 1.0, 1.0, 1.0, 1.0]$} setting, at first both tasks have an equal impact, while in the end only the main task is considered. In the setting \textit{focus-start}\footnote{$\alpha$ for 10 iterations: $\alpha = [0.0, 0.5, 0.5, 0.5, 0.5, 0.5, 0.5, 0.5, 0.5, 1.0]$}, the encoder first creates a useful representation for the helper task only and afterwards also considers the main task. As opposed to that, in \textit{focus-mid}\footnote{$\alpha$ for 10 iterations: $\alpha = [1.0, 1.0, 1.0, 0.5, 0.0, 0.25, 0.5, 1.0, 1.0, 1.0]$}, we first tune the sentence-representation only for \ac{SNLI}, adapt this representation then for the helper task and finally finetune to \ac{SNLI} again.
\item \textbf{Freeze helper-task weights:} To not encode too much logic in the helper \ac{MLP}, we freeze the weights after one iteration, dentoted as \textit{(freeze)}. Subsequent enhancements must afterwards be encoded directly in the sentence representation. 
\item \textbf{Additional weight matrix on top of sentence-encoder:} In order to achieve a strong accuracy for the helper task, more than one layer is required in the \ac{MLP}. We use a two layer \ac{MLP} for the helper task. The main task however, does not depend on the original sentence-representation anymore, but on the output of the first layer of the helper-task \ac{MLP}. We name this configuration \textit{(shared)} in the evaluation.
\item \textbf{Focus on responsible words:} Instead of using all values, encoded within the sentence-representation, we follow the same approach as described in Section §\ref{sec:understanding}, identifying which word is responsible for which dimension, and focus the network on those explicit dimensions. Thus, we consider the original pairs ($w_1$, $w_2$, \texttt{relation}) and set all dimensions within the sentence-representation to zero, if they do not arise from $w_1$. Subsequent steps remain unchanged. This is motivated by the high amount of noise from the extracted WordNet data, especially for samples within $B$. Hence, the helper-task, that should focus on the relation between $w_1$ and $w_2$, may not depend on dimensions of $w_1$, but on arbitrary other dimensions. In this particular case we do leave $B$ untouched (and do not calculate it as $B  = B \setminus (S \cup A)$), since the conflict is resolved by focusing on different parts of the sentence-representation. We refer to \textit{(max-pool)}, when applying this strategy.
\end{itemize}
\subsection{Extraction of WordNet data}
We experiment with different strategies, how to extract relevant data from WordNet, considering the lexical semantic relations, as described in Section §\ref{sec:word_relations}. We find, that by aiming for a high recall (thus considering \textit{all} senses of each word), a lot of noise is added, especially for co-hyponyms. For instance consider the word ``blue'', which is almost always used as a color. Yet, ``blue'' contains 16 different senses in total, ranging from ``sky'', ``amobarbital sodium'' or a ``family of butterflies'' to adjectives like ``aristrocratic'' or ``depressed''. This yields in antonyms like  ``lowborn'', ``cheerful'' or ``clean''. The impact is even stronger, when identifying co-hyponyms. As seen in Section §\ref{sec:wordnet}, WordNet maintains in many, but not in all cases, a very fine-grained hypernomy graph, requiring us to not only consider the next hypernym, but several hypernyms along the path. Doing so, words like ``miller'' (a type of a moth) are considered as co-hyponym, naturally in an increasing frequency, as co-hyponyms in a tree-like structure appear exorbitantly more often than antonyms or synonyms.

\subsubsection{Strategy to extract data}\label{sec:used_wordnet_extract_strategy}
Instead of the previously explained approach, we only consider the first synset to detect related synsets via the lexical relations. This does not remove noise, yet reduces it. This comes with the cost of neglecting many useful lexical semantic relations, or even using the majorily wrong sense\footnote{In Section §\ref{sec:wordnet} we showed that table (in the sense of a tabular visualization) is the first sense of table (as opposed to the sense of a furniture). Yet, in \ac{SNLI} we especially expect the second sense to be useful.}. The main motivation is, that we lack of automatic evaluation methods for the extracted data w.r.t. \ac{SNLI}, conflicting with the time constraints for the remainder of the work. Since previous experiments\footnote{Conducted by Vered Shwartz and not part of this work.} focused on a high recall and did not improve the performance, we now aim for a high precision of the extracted data. Note, that the WordNet baseline, as defined in Section §\ref{sec:additional_snli_set} did not suffer from the same problem, since all word-pairs ($w_p$, $w_h$) already are known to have a meaningful relation (based on their creation process), thus the extracted relation between both words most likely is valid. The problem only occurs, as we intend to extract words the other way around, by knowing the relation.

\subsubsection{Final extracted data}
We consider all lemmata within the first synset of $w$ as synonyms. Hypernyms of $w$ are considered up to an edge length of 5. For co-hyponyms, we consider all hyponyms of hypernyms of $w$, both bound to an edge length of three, only if they are not also a hypernym of $w$. 
\begin{table}[tph!]
\centering
\begin{tabular}{cc|cc}
\multicolumn{2}{c}{\textbf{$\mathbf{A}$ (entailed)}} & \multicolumn{2}{c}{\textbf{$\mathbf{B}$ (contradicting)}} \\
\textbf{$\mathbf{w_1}$} & \textbf{$\mathbf{w_2}$} & \textbf{$\mathbf{w_1}$} & \textbf{$\mathbf{w_2}$} \\
\toprule
oppose & content & Trojan & Iraqi\\
pug & dog & waffle & Cheesecake\\
reward & rewarding & five & trio \\
townspeople & town & inferno & radius \\
pop & bulge & conditioner & aerobics\\
permit & permit & killers & party\\
frolics & play & hiding & processes\\
leading & ahead & chapel & synagogue\\
commitment & sincerity & Villages & crossroads\\
Wool & material & saloon & Minivan\\
\bottomrule
\end{tabular}
\caption{Examples of extracted word-pairs ($w_1$,$w_2$) for both categories, being represented by the sentence containing $w_1$ ( thus $A$) or not (thus $B$).}
\label{tab:examples_extracted_wn}
\end{table}
We also consider \textit{part-meronyms}, that are a hyponym of ``location'' of any distance. Since the interpretation of meronyms is not trivial w.r.t. entailment, we thus only consider it in the context of locations and assume that a meronym entails its holonym\footnote{As in ``John is in Berlin.'' $\Rightarrow$ ``John is in Germany.'', with ``Berlin'' \textit{part-of} ``Germany''}. We generate a total of 686,265 word pairs with their entailment interpretation into either $A$ or $B$, precisely $|A|=104,550$ and $|B|=581,715$. Note that these include the word itself like ($w_1$,$w_1$, $A$), if lexical semantic relations for a $w^s$ are found. All words appear within the \ac{SNLI} dataset and thus can be useful. We show random samples of ten pairs for each class respectively in Table \ref{tab:examples_extracted_wn}.
Obviously, the data is still not entirely clean, however one can at least identify, why several word-pairs are within $A$ or $B$. Applying the extracted data on sentences within \ac{SNLI} train data, yields to an average of 44.6 samples from $A$ and 300.0 samples from $B$ for the helper-task for each single sentence, $p$ or $h$.
\subsection{Evaluation}
We evaluate all previously explained experiments within this section.
\subsubsection{Integrate WordNet using embeddings}
Table \ref{tab:eval_embeddings_added} shows the performance of the concatenated word-embeddings together with the performance of the unchanged Residual-Stacked Encoder\textsuperscript{$\dagger$}. The upper part contains embeddings that have been newly created or changed to contain other than (only) distributional information, the lower part shows the concatenated hypernyms, using the original (but fine-tuned during training) distributional representations.
\begin{table}[tph!]
\centering
\begin{tabular}{rc|cc|cc}
\textbf{Additional Embeddings} &\textbf{Dimensions} & \textbf{\ac{SNLI} test} & \textbf{$\Delta$} & \textbf{New test} & \textbf{$\Delta$}\\
\toprule
Attract-Repel \citep{ruckle2018concatenated} & 300D & 85.4\% & $-$0.4 & 58.3\% & $-$0.9 \\
Trained-All (cross-entropy) & 50D & 85.4\% & $-$0.4 & 59.2\% & $\pm$0 \\
Trained-Syn-Ant (eucledian) & 20D & 85.4\% & $-$0.4 &57.8\% & $-$1.4 \\
Trained-Syn-Cohyp (eucledian)& 20D & 85.5\% & $-$0.3 &55.7\% & $-$3.5 \\
Trained-Syn-Cohyp-Ant (eucledian)& 20D+20D & 85.7\% & $-$0.1 & 56.6\% &  $-$2.6 \\
\midrule
Hypernyms-1 & 300D  & 85.2\% & $-$0.6 & 54.8\% & $-$4.4 \\
Hypernyms-3 & 300D  & 85.3\% & $-$0.5 & 60.8\% & $+$1.6 \\
Hypernyms-5 & 300D  & 85.4\% & $-$0.4 & 66.4\% & $+$7.2 \\
\midrule
\textbf{Residual-Stacked Encoder\textsuperscript{$\dagger$}} & $-$ & 85.8\% &$\pm$0 & 59.2\% & $\pm$0\\
\bottomrule
\end{tabular}
\caption{Evaluation of experiments with additional information in the word-representations, compared to the Residual-Stacked Encoder\textsuperscript{$\dagger$} (bottom).}
\label{tab:eval_embeddings_added}
\end{table}
All methods slightly decrease in terms of accuracy for the original \ac{SNLI} test set, even though only additional information is added. This however is not significant and most likely stems the parameters being highly tuned towards \ac{SNLI} for the original model. Since we do not intend to increase the performance by a small margin coming from hyperparameter settings, we do not fine-tune these. Also on the new test set, most approaches do not show major differences to the original model. Only the concatenation of hypernyms shows with an increasing amount improvements, which is even stronger than the accuracy achieved by \ac{ESIM} \citep{chen2017enhanced}. 
\subsubsection{Integrate WordNet using multitask-learning}\label{sec:eval_mt}
We depict the results of experiments using multitask-learning, as described in Section \ref{sec:mt_learning_intro}, in Table \ref{tab:mt_evaluation}.
\begin{table}[tph!]
\centering
\resizebox{\textwidth}{!}{%
\begin{tabular}{rccc|cr|cr}
\textbf{Helper-task MLP} & $\alpha$ & \textbf{dropout} & \specialcellc{\textbf{Re-sample}\\\textbf{less frequent}} & \textbf{\ac{SNLI} test} & $\Delta$ & \textbf{New test} & $\Delta$ \\
\toprule
2 $\times$ 100D & 0.5 & $-$ & yes & 84.6\% & $-$1.2 & 47.2\% & $-$12.0 \\
2 $\times$ 600D & 0.5 & $-$ & yes & 85.2\% & $-$0.6 & 59.9\% & $+$0.7 \\
2 $\times$ 300D & 0.5 & yes & yes & 84.8\% & $-$1.0 & 52.5\% & $-$6.7 \\
2 $\times$ 600D & 0.5 & yes & yes & 84.8\% & $-$1.0 & 51.8\% & $-$7.4 \\
2 $\times$ 300D & 0.5 & yes & $-$ & 85.2\% & $-$0.6 & 48.7\% & $-$10.5 \\
2 $\times$ 600D & 0.5 & yes & $-$ & 85.0\% & $-$0.8 & 57.7\% & $-$1.5 \\
2 $\times$ 300D & 0.75 & yes & yes & 85.3\% & $-$0.5 & 61.0\% & $+$1.8 \\
2 $\times$ 300D & 0.75 & yes & $-$ & 85.7\% & $-$0.1 & 58.9\% & $-$0.3 \\
\midrule
2 $\times$ 300D & \texttt{finetune} & yes & yes & 84.9\% & $-$0.9 & 51.5\% & $-$7.7 \\
2 $\times$ 300D & \texttt{finetune} & $-$ & yes & 84.5\% & $-$1.3 & 52.3\% & $-$6.9 \\
2 $\times$ 600D & \texttt{focus-start} & $-$ & yes & 85.4\% & $-$0.4 & 46.9\% & $-$12.3 \\
2 $\times$ 600D & \texttt{focus-mid} & $-$ & yes & 85.4\% & $-$0.4 & 59.6\% & $+$0.4 \\
\midrule
2 $\times$ 300D (freeze) & 0.75 & $-$ & yes & 85.3\% & $-$0.5 & 53.2\% & $-$6.0 \\
2 $\times$ 300D (freeze) & 0.5 & $-$ & yes & 84.7\% & $-$1.1 & 44.8\% & $-$14.4 \\
2 $\times$ 800D (shared) & 0.5 & yes & yes & 84.4\% & $-$1.4 & 42.2\% & $-$17.0 \\
2 $\times$ 600D (shared) & 0.75 & yes & yes & 84.6\% & $-$1.2 & 42.3\% & $-$16.9 \\
2 $\times$ 400D (shared) & 0.5 & yes & yes & 83.9\% & $-$1.9 & 34.3\% & $-$24.9 \\
\midrule
2 $\times$ 300D (max-pool) & 0.75 & yes & yes & 84.8\% & $-$1.0 & 57.8\% & $-$1.4 \\
\midrule
\textbf{Residual-Stacked Encoder\textsuperscript{$\dagger$}} &$-$&$-$&$-$&85.8\%&$\pm$0&59.2\%& $\pm$0 \\
\bottomrule

\end{tabular}}
\caption{Evaluation of experiments using multitask-learning, compared with the Residual-Stacked Encoder\textsuperscript{$\dagger$.}}
\label{tab:mt_evaluation}
\end{table}
The first column shows the dimensions of the helper-task \ac{MLP}. We only used two layer \ac{MLP}s with the specified dimensions, as previous experiments showed that smaller networks have already problems reaching a high accuracy of the helper-task. The presented networks all solve this task with an accuracy of $> 90 \%$ (dev). Similarily, we observe that, reducing  $\alpha$, thus increasing the impact of the helper-task, in most cases reduces the performance on the main task. Generally, reducing the impact of the helper-task, by increasing the complexity of the helper-task \ac{MLP}, omitting dropout or by increasing $\alpha$, the performance drops less (or slightly improves). Subsequently, experiments with a very strong impact of the helper task, as sharing a layer of its \ac{MLP} or freezing its weights, result in a very poor performance. The dynamic adaptions of $\alpha$ in the second part of the table only show comparable results to the original model, if the helper-task is considered for a few iterations only. By taking max-pooling information into account, the performance decreases slightly. While this potentially should lay the focus on relevant words, we neglect the fact, that certain attributes may be shifted, due to the context implementing nature of \ac{LSTM}s. Thus, for ``a happy child'' the information for \textit{being happy} might not only be present within ``happy'', but also within ``child'', as the first describes the second word. This may result in ``child'' having a higher value within the relevant dimension. While this dimension for ``happy'' may still be relatively high, in our experiment we would neglect this dimension completely (by setting it to zero), when predicting ``happy'' in the helper-task. Thus a more sophisticated approach might be, to consider the output vectors after each timestep for the according word directly, instead of masking the final sentence representation. Re-sampling the helper task such that $|A| = |B|$ seems superior to not-resampling.

\subsection{Analysis}
We analyse selected experiments of both approaches in this section.

\subsubsection{Integrate WordNet using embeddings}
\begin{table}[tph!]
\centering
\begin{tabular}{rcc|cr|cr}
\textbf{Category}  & \textbf{Amount}& \specialcellc{\textbf{Residual-Stacked}\\\textbf{Encoder}\textsuperscript{$\dagger$}} & \specialcellc{\textbf{Attract-}\\\textbf{Repel}} & $\Delta$ &  \textbf{Hypernyms-5} & $\Delta$ \\
\toprule
antonyms & 1,147 & 51.0\% & 53.8\% & $+$2.8 & 74.2\% & $+$23.2 \\
cardinals & 759 & 20.3\% & 19.1\% & $-$1.2 & 15.3\% & $-$5.0 \\
nationalities & 755 & 44.2\% & 31.1\% & $-$13.1 & 56.7\% & $+$12.5 \\
drinks & 731 & 89.7\% & 93.3\% & $+$3.6 & 72.7\% & $-$17.0 \\
antonyms(WN) & 706 & 63.2\% & 60.8\% & $-$2.4 & 71.2\% & $+$8.0 \\
colors & 699 & 90.8\% & 88.1\% & $-$2.7 & 95.2\% & $+$7.1 \\
ordinals & 663 & 3.0\% & 8.8\% & $+$5.8 & 6.2\% & $+$3.2 \\
countries & 613 & 75.4\% & 44.1\% & $-$31.3 & 81.6\% & $+$6.2 \\
rooms & 595 & 73.1\% & 77.0\% & $+$3.9 & 75.0\% & $+$1.9\\
materials & 397 & 80.4\% & 77.6\% & $-$2.8 & 85.1\% & $+$4.7 \\
vegetables & 109 & 40.4\% & 41.3\% & $+$0.9 & 41.3\% & $+$0.9 \\
instruments & 65 & 96.9\% & 98.5\% & $+$1.6 & 98.5\% & $+$1.6 \\
planets & 60 & 61.7\% & 31.7\% & $-$30.0 & 38.3\% & $-$23.4 \\
\midrule
synonyms & 894 & 73.9\% & 92.6\% & $+$18.7 & 74.9\% & $+$1.0 \\
\midrule
total & 8,193 & 59.2\% & 58.3\% & $-$0.9 & 66.4\% & $+$7.2 \\
\bottomrule
\end{tabular}
\caption{Accuracy per category for concatenated embeddings using Attract-Repel or Hyponyms-5.}
\label{tab:detail_added_embds}
\end{table}
Table \ref{tab:detail_added_embds} shows the accuracy per category for \textit{Attract-Repel}, as being the most sophisticated word-representations with additional non-distributional information and \textit{Hypernyms-5}, as the best achieved result. Comparing the model using concatenated embeddings from Attract-Repel with the Residual-Stacked Encoder\textsuperscript{$\dagger$} indicates, that different features are considered relevant by the netwrok. Most of the differences within the categories seem arbitrary, arising most likely from this different feature selection, as the original information would still be accessable to the model. Synonyms are strongly improved compared to the original model, which can easily be explained due to an even higher word-vector similarity from Attract-Repel post-processed representations. On the other hand, both antonym groups show no substantial improvement. The antonyms derived from WordNet show an even worse accuracy than before. Overall, the performance gained using these embeddings with all deviations within different categories is highly similar to the ones achieved using multitask learning (next Section). Since the added word-vectors encode differences for most antonyms, but do not leverage them, we assume this stems from a lack of representative data within \ac{SNLI} data. Naturally, if the model does not depend on those differences during training, it will not learn to consider them in the prediction process.

\paragraph*{Impact of hypernyms}
Looking at the concatenation with the hypernyms for each word, the increase in performance looks much more stable. Ignoring \textit{planets}, which are highly noise-sensitive due to their limited size, only two categories do not improve over the baseline. Synonyms also (like mutually exclusive words) share similar hypernyms, the model must learn this differentiation. Looking at the categorical evaluation, the performance of synonyms remains similar to the original evaluated model, the Residual-Stacked Encoder\textsuperscript{$\dagger$}, showing that it does not suffer from added hypernyms in this aspect. Yet, we observe that adding hypernyms does not help in identifying synonyms, which is based on the results in Section §\ref{sec:additional_snli_set} a relatively easy to classify category. Especially the overall improvement for contradicting examples is interesting and we closer examine the impact of the hypernyms for these samples. To focus on the actual impact of the new information, we exclude all samples that are predicted identically as by the Residual-Stacked Encoder\textsuperscript{$\dagger$}, thus exclude samples that are correctly classified based on memorizing word-relations. 
\begin{figure}[tph!]
\centering
	\includegraphics[totalheight=5.5cm]{fig/analyse_hypern5.png}
	\caption{Comparison of contradicting samples (different w.r.t. correctness from Residual-Stacked Encoder\textsuperscript{$\dagger$}). for Hypernyms-5, by the amount of shared hypernym embeddings.}
	\label{fig:analyse_hypern5}
\end{figure}
Figure \ref{fig:analyse_hypern5} visualizes the impact of the hypernym embeddings by comparing how many hypernym embeddings are shared for each word-pair($w_p$,$w_h$). In total, 1378 contradicting samples are classified differently from the original model, 984 are now classified correctly as contradiction (blue), 394 samples are now misclassified (red). The x-axis shows the percentage $x_{w_p,w_h}$ of shared hypernym lemmata between $w_p$ and $w_h$, calculated as 
\begin{equation}
x_{w_p,w_h} = \frac{|H_{w_p} \cap H_{w_h}|}{|H_{w_p} \cup H_{w_h}|}
\end{equation}
with $H_{w_p}$ and $H_{w_h}$ being the sets lemmata of hypernyms from $w_p$ and $w_h$ respectively, gathered using the same method as to create the embeddings. The y-axis displays the proportional amount of each group. Contrarily to our intention, the model does not seem to use the additional vectors to identify co-hyponyms. Instead, especially if only few hypernyms are identical (fewer indicators for co-hyponym), more samples are re-predicted correctly, whereas a higher similarity of the hypernyms leads to a higher amount of incorrect predictions. The fact that those ``unrelated'' words are an indicator for contradiction highly correlates with the data seen during training. Due to the creation process of \ac{SNLI} (in addition to to frequently changed words as identified by \cite{gururangan2018annotation}) many contradicting samples describe very unrelated scenarios, yielding in contradiction based on the event-coreference, not because they contain related, but contradicting, words \citep{dasgupta2018evaluating}. Subsequently, unrelatedness of words serves as a good indicator for contradiction. Yet, looking at categories individually, we observe that at least for some of them, the model learned the tendency of leveraging from the added information in the intended way. Correct classified cardinals in general all share $\geq 0.7$ \% of their hypernym lemmata, thus are very similar to each other. A total of 200 cardinal samples are either in the red or blue group visualized. Of those, 143 (71.5\%) have been classified correctly with the concatenated hypernyms and only 57 (28.5\%) are misclassified compared to the original predictions. Similar, but less strongly, are 63 (61.8\%) countries with $\geq 0.7$\% shared hypernym lemmata classified correctly, and only 39 (38.2\%) incorrectly. Opposed to these categories, for nationalities or antonyms, the improvements compared to the base model arises from unrelated word-pairs (in terms of their shared hypernym lemmata). In both cases, the majority of different re-prediction stems from word-pairs sharing no lemma within their hypernyms. Only looking at those word-pairs, sharing not a single word-vector for their hypernyms, 109 (88.6\%) of 123 nationalities\footnote{Other \textbf{mispredictions}: 2$\times$ with 0.8 shared, 3$\times$ with 0.3 shared. Other \textbf{correct} predictions: 5$\times$ with 0.1$-$0.5 shared, 16$\times$ with 0.7$-$0.9 shared} and 278 (90.3\%) of 308 antonyms\footnote{Other \textbf{mispredictions}: 52 $\times$ 0.5$-$0.9 shared. Other \textbf{correct} predictions: 5$\times$ with 0.2$-$0.5 shared, 41 with 0.5$-$0.9 shared.} are re-predicted correctly. In total, the model seems to slightly benefit from the added information in some cases in the intended way, in the majority of cases however, the improvement in performance seems to stem from another frequently occuring pattern in \ac{SNLI}, namely unrelated hypotheses, rather than improving the general \ac{NLU}. 

\subsubsection{Integrate WordNet using multitask-learning}
For the analysis of multitask-learning we select the best performing model on the new test-set using a 300-dimensional helper task \ac{MLP} with $\alpha=0.75$ with re-sampling and dropout, denoted as \textit{300D-0.75 STD} and the comparable model with with a stronger impact of the helper-task with $\alpha=0.5$, also 300 dimensions, re-sampling and dropout, named as \textit{300D-0.5 STD}. Additionally, we select the model using the max-pooled information, referred to as \textit{300D-0.75 max-pool}, as it puts the focus on the actual word relations and performs comparably with the Residual-Stacked Encoder\textsuperscript{$\dagger$}. The accuracy per category of these models is displayed in Table \ref{tab:categories_mt}.
\begin{table}[tph!]
\centering
\begin{tabular}{rc|cr|cr|cr}
\textbf{Category}  & \textbf{Amount}& \specialcellc{\textbf{300D-0.75}\\\textbf{STD}} & $\Delta$ &  \specialcellc{\textbf{300D-0.5}\\\textbf{STD}} & $\Delta$ & \specialcellc{\textbf{300D-0.75}\\\textbf{max-pool}} & $\Delta$\\
\toprule
antonyms & 1,147 & 39.4\% & $-$11.6 & 35.6\% & $-$15.5 & 44.9\% & $-$6.1 \\
cardinals & 759 & 40.1\% & $+$19.8 & 26.2\% & $+$5.9  & 33.3\% & $+$13.0\\
nationalities & 755 & 52.5\% & $+$8.3 & 32.1\% & $-$12.1 & 25.3\%  & $-$18.9\\
drinks & 731 & 75.9\% & $-$13.8 & 67.4\% & $-$22.3 & 82.5\% & $-$7.2\\
antonyms(WN) & 706 & 60.5\% & $-$2.7 & 56.2\% & $-$7.0 & 56.5\% & $-$6.7\\
colors & 699 & 90.6\% & $-$0.2 & 86.3\% & $-$4.6 & 88.0\% & $-$2.8\\
ordinals & 663 & 3.0\% & $\pm$0 & 2.7\% & $-$0.3 & 3.2\% & $+$0.2\\
countries & 613 & 69.8\% & $-$5.6 & 40.6\% & $-$34.8 & 69.7\% & $-$5.7\\
rooms & 595 & 74.1\% & $+$1.0 & 65.2\% & $-$7.9 & 74.6\% & $+$1.5\\
materials & 397 & 85.9\% & $+$5.5 & 79.8\% & $-$0.6 & 72.3\% & $-$8.1\\
vegetables & 109 & 41.3\% & $+$0.9 & 44.0\% & $+$3.6 & 32.1\% & $-$8.3\\
instruments & 65 & 95.4\% & $-$1.2 & 93.8\% & $-$3.1 & 95.4\% & $-$1.5\\
planets & 60 & 35.0\% & $-$26.7 & 26.7\% & $-$35.0 & 58.3\% & $-$3.4\\
\midrule
synonyms & 894 & 97.6\% & $+$23.7 & 96.1\% & $+$22.2 & 94.4\% & $+$19.5\\
\midrule
total & 8,193 & 61.0\% & $+$1.8 & 52.5\% & $-$6.7 & 57.8\% & $-$1.4\\
\bottomrule
\end{tabular}
\caption{Accuracy per category for selected models using multitask-learning.}
\label{tab:categories_mt}
\end{table}
It can be seen that the majority of all contradicting categories performs worse than the base-model. Only the synonyms highly leverage from this method and reassemble the performance reached by attention-based models in Section §\ref{sec:additional_snli_set}. We observe this phenomenon on synonyms for all 17 evaluated models of Section §\ref{sec:eval_mt}. The extracted synonyms from WordNet are much less noisy than co-hyponyms, indicating that the helper-task is more likely to consider the relevant dimensions coming from the synonym for its prediction. However, as seen by the model taking max-pooling information into account, the performance on contradicting samples is still very poor. Only two categories seem to be improved on a relatively constant basis. Cardinals improve in 7/17 approaches, mostly by more 10 ten points in accuracy, vegetables improve in 9/17 approaches, with a maximum of 9.1\% increase. Especially the drop within the second model for countries is severe, yet not only present within this experiment. Not a single model of our evaluations superceeded the original model for countries, 8/16 decreased by more than 35 points in accuracy, another 2 experiments by more than 10 points. Since the other models of Section §\ref{sec:additional_snli_set} achieve similar results, and Residual-Stacked Encoder\textsuperscript{$\dagger$} is highly aligned with its hyperparameters to Residual-Stacked Encoder\textsuperscript{$\Diamond$}, we assume the high performance arises from correctly picked features by chance, rather than stemming from the model's architecture. 

\paragraph*{Impact of the selected data}
Due to the restricted method of extracting word-pairs, defined in Section §\ref{sec:used_wordnet_extract_strategy}, the upper bound, that can be achieved using this information drops. Thus, in the following analysis we only look at samples, that could have been classified correctly, based on this extracted data from WordNet. The results are depicted in Table \ref{tab:eval_mt_data}. We report the absolute amount of samples together with the percentage, compared to the original size of each category. The categories drinks (3), instruments (11), vegetables (20), materials (35) and planets (37) are aggregated, due to insufficient amount of samples for any representative conclusions. All $\Delta$ show the difference to the Residual-Stacked Encoder\textsuperscript{$\dagger$} on the same data, instead of comparing them with the same model using the full data.
\begin{table}[tph!]
\centering
\resizebox{\textwidth}{!}{%
\begin{tabular}{rccc|cr|cr|cr}
&\multicolumn{2}{c}{\textbf{Amount}}& \specialcellc{\textbf{Residual-Stacked}\\\textbf{Encoder\textsuperscript{$\dagger$}}} & \multicolumn{2}{c}{\textbf{300D-0.75 STD}} & \multicolumn{2}{c}{\textbf{300D-0.5 STD}} & \multicolumn{2}{c}{\specialcellc{\textbf{300D-0.75}\\\textbf{max-pool}}}\\
\textbf{Category} & \# & \% & Acc. & Acc.& $\Delta$ & Acc.& $\Delta$ & Acc.& $\Delta$\\
\toprule
antonyms & 885 & 77.2\% &51.3\% &37.7\%&$-$13.6&36.5\% &$-$14.8  &46.6\% &$-$4.7\\
cardinals &496 &65.3\%&21.6\% &41.2\%&$+$19.6&29.6\% &$+$8.0 &34.7\% &$+$13.1\\
countries & 471&76.8\% &75.6\% &66.2\%&$-$9.4&33.3\% &$-$42.3 &68.2\% &$-$7.4\\
nationalities &427 & 56.6\% &43.4\% &59.3\%&$+$15.9&32.8\% &$-$10.6 &32.1\% &$-$11.3\\
antonyms(WN) &379 & 53.7\%&77.6\% &76.0\%&$-$1.6&70.8\% &$-$6.8 &69.7\% &$-$7.9\\
colors & 312& 44.6\%&95.8\% &95.8\%&$\pm$0&93.3\% &$-$2.5 &93.3\% &$-$2.5\\
ordinals &263 &39.7\% &6.5\% &7.2\%&$+$0.7&6.5\% &$\pm$0 &6.5\% &$\pm$0\\
rooms &213 &35.8\% &94.8\% &85.4\%&$-$9.4&76.1\% &$-$18.7 &95.3\% &$+$0.5\\
\textit{other} & 106 & 7.8\% & 33.0\% &48.0\%&$+$15.0&52.8\% &$+$19.8 &33.0\% &$\pm$0\\
\midrule
synonyms & 385& 43.1\% &98.2\% &100.0\%&$+$1.8&99.7\%&$+$1.5  & 100\%& $+$1.8\\
\midrule
\textbf{Total} & 3,937 & 48.1\% & 60.1\%&59.5\%&$-$0.6&49.7\% &$-$10.4 &57.5\% &$-$2.6 \\
\bottomrule
\end{tabular}}
\caption{Accuracy per category of three selected multitask-learning experiments compared with Residual-Stacked Encoder\textsuperscript{$\dagger$} on samples covered by extracted word-pairs.}
\label{tab:eval_mt_data}
\end{table}
The 100\% accuracy on synonyms is not very surprising, given the fact that only synonym examples are included, if they can be explained using the extracted data. Thus, samples of this category that have another label than entailment, due to the usage in context, are excluded. The performance gain in the aggregated \textit{other} category mostly stems from materials. All of the multitask experiments, generally perform worse than the base model without multitask-learning. If we compare the overall performance of each model on this subset of data with the performance achieved over all data, we observe that only the original Residual-Stacked Encoder\textsuperscript{$\dagger$} improved in accuracy, while other models perform worse than before. Subsequently, they perform slightly better on the other half of the data, that cannot be explained using the fused information. \cite{chen2017natural} show with \ac{KIM} and a total of 5,425,426 extracted word pairs\footnote{Note, that they align words of $p$ and $h$ to identify their relation, thus they may not suffer that much from arbitrary word relations, extracted from WordNet.}, being crucially more than ours, that the network especially leverages, if $\geq$ 40\% of the external knowledge is used. Compared too that, our experiments may indeed suffer from limited coverage, especially since \cite{chen2017natural} directly encode the lexical relations and we still rely on the \ac{MLP} to identify them, indirectly encoded within the sentence-representation. Yet, it could have been expected, that models would have an advantage on this subset of data either way. Since they obviously fail leverage from the fused information, sufficient for all those samples, the problem seems to rather be the method than the data.

\paragraph*{Impact on the sentence representation}
Looking back to the original intention, of fine-tuning embeddings and ensuring those differences would be present within the sentence-representation, we take a closer look at this aspect. Following the results from the previous step we can only see, that the multitask-learning is not heelpful for the final prediction, which can have several reasons: Either the helper-task did not manage to encode the relevant differences into the sentence-representation, or it did, but the model failed to use them. We use the same technique as introduced in Section §\ref{sec:approach_general_alignment_understanding}, visualizing the alignment of the sentence-representations of $p$ and $h$. We only calculate the averaged counts per dimension for the same subset of the data from the previous section, but only looking at contradicting samples (3511 in total).
\begin{figure}[tph!]
\centering
	\includegraphics[totalheight=7cm]{fig/base_correct_incorrect_c.png}
	\caption{Aligned $p$ and $h$ for all contradicting sampes, covered by the fused WordNet information, correctly predicted (left) or mis-predicted (right).}
	\label{fig:base_correct_incorrect_c}
\end{figure}
Figure \ref{fig:base_correct_incorrect_c} shows the aligned sentences for the Residual-Stacked Encoder\textsuperscript{$\dagger$}, differentiating between the 1954 correctly classified samples and 1557 misclassified samples. As expected, the majority of dimensions have the same value, arising from the high lexical overlap and the fact that word-pairs are selected to be replacable in context, thus will have similar embeddings. Even though the network structure, dimensions and performance, compared to the analysed model Shortcut-Stacked Encoder\textsuperscript{$\dagger$}, slightly changed, the results gained from this section still seem applicable. Due to the similarity of both sentences, only few dimensions differ. We observe more of these differing high dimensions for the correctly classified samples, in many cases more than twice the amount compared to the misclassified samples. Knowing that the Residual-Stacked Encoder\textsuperscript{$\dagger$} most likely follows the same principles as identified in Section §\ref{sec:understanding}, we now look at the sentence-representations gained from multitask-learning (for the same data). Figure \ref{fig:300d75_correct_incorrect_c} shows the aligned sentence-representations (correct and misclassified) for the best multitask-learned model \textit{300D-0.75 ST}.
\begin{figure}[tph!]
\centering
	\includegraphics[totalheight=7cm]{fig/300d75_correct_incorrect_c.png}
	\caption{Aligned $p$ and $h$, correctly predicted (left) and mis-predicted (right) for multitask-learned \textit{300D-0.75 STD}.}
	\label{fig:300d75_correct_incorrect_c}
\end{figure}
Comparing these visualizations with the orignal model, clearly both, the mispredicted and correctly predicted sentence-representations show a higher amount of different high-valued dimensions. In line with our other observations, this is especially visible for the correctly classified samples. While the Residual-Stacked Encoder\textsuperscript{$\dagger$}, as well as the Shortcut-Stacked Encoder\textsuperscript{$\dagger$} showed the majority of dimensions within the positive values area, in this case a huge amount of samples are also within the negative area. We do not start another dimension-wise analysis for this model and leave it open for interpretation. Since this phenomenon stems from the helper-task, one possible explanation is, that low values indicate the absence of specific words. Having a large lexical overlap, $p$ and $h$ lack the identical words or meaning (coming from $B$) for the majority of words. Thus the dominant symmetry for the negative values may arise. Even though the main-task may still optimize to deal with negative values or interpret the absence of information within a dimension not aligned with zero but another value (as opposed to the original model), this breaks the same behaviour that was naturally learned by both models without multitask-learning. Similarily, \cite{mou2015natural} showed, that even though the model is able to learn element-wise difference and product (which both work intuitively well with absence of information being encoded close to zero) by themselves, using it in the feature concatenation helps the performance. Yet, we do not further investiate this phenomenon and it may not even be harmful to the final prediction. Similar results are shown for the other two experiments, picked for the analysis part\footnote{We did not conduct similar analysis for the remaining models.}. Figure \ref{fig:masking_e_c} shows all contradicting and entailing samples of the selected data, regardless of the prediction of the model, encoded by \textit{300D-0.75 max-pool}.
\begin{figure}[tph!]
\centering
	\includegraphics[totalheight=7cm]{fig/masking_e_c.png}
	\caption{Aligned $p$ and $h$, of contradicting (left) and entailing (right) samples for multitask-learned \textit{300D-0.75 max-pool}.}
	\label{fig:masking_e_c}
\end{figure}
This seems to be a bit more fuzzy, compared to the other two multi-tasking experiments. Neglecting the high amount of negative valued dimensions, this is exactly the kind of representations we were aiming for, clearly encoding the same information for entailing examples, while also encoding differences for the contradicting samples\footnote{This also can be seen for the other two evaluated models.}. As seen in Section §\ref{sec:understanding_align_neutral_contr}, this however is not sufficient for the model to predict contradiction, as distinct information is also present for neutral samples. We conclude, that we managed to shift the sentence-representation in a way, that it encodes differences for antonyms and co-hyponyms stronger than before. However the classifying \ac{MLP} lacks to leverage from those in the intended way. 
\subsection{Summarizing experiments to incorporate WordNet}
We encountered the challenge of gaining high quality data out of WordNet. Even though lexical resources contain a huge amount of information, it is required to put more effort into the extraction of the data. In our case, we applied a simple strategy to increase the precision of the exracted data, yet at the expense of a lot of valuable information. We have shown, that adding additional information to the network inputs (the embeddings) may have a good impact in some cases, however we still depend on the train data to rely on those additional information, and the model to identify and consider them as relevant features. This is somewhat challenging, since some of these information might be less frequent in the data, while highly represented arbitrary patterns within the train data are more important w.r.t. the optimmization function. We tackled this problem using multitask-learning and have successfully transferred word similarities, based on the fused WordNet information, into the sentence-representation, yet the final \ac{MLP} failed to consider them in the desired way. This basically breaks down to the same problem, that the changed encodings are not considered relevant w.r.t. the train data.
\paragraph*{Comparing to \ac{KIM}}
As opposed to the sentence-encoding Residual-Stacked Encoder, \ac{KIM} \citep{chen-EtAl:2017b:natural} uses inter-sentence attention and identifies directly the WordNet relation for words within $p$ and $h$, while we only use a indirect way to encode this. Their approach however, seems very elegant, considering the heavy influence of the train data, whether features are considered relevant or not. \cite{gururangan2018annotation} identified the \ac{SNLI}-specific patterns not as a problem because they are incorrect, but because they are highly dominant, leading to oversimplified solutions only based on those. Thus, a model has a good chance of being correct to classify a sample as contradicting if a ``cat'' is in $h$ and a ``dog'' is in $p$. This is not hard to learn for a neural network, if seen in a large number of times. However, instead of memorizing this specific word-pair, knowing that both words are close hyponyms of the same hypernym (``pet'') also serves a simple and effective feature for the same problem. By assigning each lexical relation (quantified by their distance) holding between two words, onto one specific dimension, \cite{chen-EtAl:2017b:natural} made this simple and powerful feature easily accessible, and also show that especially the information for co-hyponomy is beneficial for \ac{SNLI}. Additionally to having this \textit{meaningful} decision criteria for one specific word-pair, the model can apply the same strategy on other co-hyponyms and indeed achieve a better generalization ability. As opposed to their strategy, our indirect way to encode differences still relied on the importance of those encoded differences for the train data. Even if the same lexical relation can be inferred from the new representations (which would be the best case, but most likely is not that simple), this may depend on different dimensions for different words, as opposed to always being indicated in the same way. We conclude that WordNet information must be fused in a way, that is generally applicable and can easily be identified by the model, in order to overcome certain patterns and be more useful than memorizing those.\newpage\cleardoublepage
\section{Conclusion and future work}
In this work we showed at the sampe of \ac{SNLI}, that even though being intended to improve \ac{NLU}, the high performance gained by state-of-the-art models for \ac{NLI} does not reflect the actual \ac{NLU} capabilities. All models performed significantly lower on our adversarial dataset, derived from the original train-set but excluded  \ac{SNLI}-specific patterns. Even though \ac{KIM} performed quite well, future work may find ways to integrate more data or methods to deal with lexical inferences in context to improve the performance, which has not yet met the limit. We attempted to improvce the performance on this new dataset using WordNet information for a sentence-encoding model. Here, we leveraged from the max-pooled sentence-encoding and showed that this can be used to understand the dimensional values and sentence-representations, generated by the model. In addition to showing that this information can indeed be used to change the representations in a meaningful way, our inisghts gained here, have also shown to be useful for an intrinsic analysis of the sentence-representations of the experiments incorporating WordNet. Future work can develop deper insights on these representations on a broader range of data and models, to enable more meaningful analysis or even adaptions. This may lead to better defined helper-tasks for multitask-learning approaches, such that the dimensions are adapted under consideration of the original sentence encoding scheme. Finally we evaluated in our approaches to incorporate WordNet, that this should be done in a general and very easy to identify way, in order to be relevant enough to overcome diminant patterns in a dataset. Especially for sentence-encoding models this seems very challenging at the moment, future work may leverage from structures like memory networks \citep{sukhbaatar2015end}, to do so.

\section*{Acknowledgements}
Special Thanks to Yoav Goldberg and Vered Shwartz, for their continuous support and guiding for my master thesis, while still giving e the freedom to pursue my own ideas. I am also thankful for Andreas Rücklé and Andreas Hanselowski for supervising me from Germany, and even though the far distance assited with valuable advice and discussions. The possibility for me to conduct my thesis at Bar-Ilan was enabled by Iryna Gurevych an Yoav Goldberg within a very short time and without complications, which I am also very thankful for. Finally I want to thank the remainder of the group at Bar-Ilan for a great stay. Additionally, Corinna Neureither double checked my final thesis, and is to blame for any remaining spelling errors.\newpage\cleardoublepage

	
\bibliography{bib/literature}\newpage\cleardoublepage


\listoffigures\newpage\cleardoublepage

\listoftables\newpage\cleardoublepage

\end{document}
