\section{Understanding Shortcut-Stacked-Encoder}\label{sec:understanding}
In this section we analyse the sentence-representations of Shortcut-Stacked Encoder\textsuperscript{$\dagger$}, by visualizing how they encode natural language sentences coming from \ac{SNLI} (Section §\ref{sec:insights_sent_repr}) and how they use the created sentence representations (Section §\ref{sec:insights_sent_alignment}). Additionally we show experiments, underlining the presented insights.
\subsection{Motivation}
The major downside of neural networks is the lack of interpretability \citep{goldberg2017Apr}. Thus, their capabilities and decision criteria can only be estimated by finding meaningful evidence for their failures or sucesses on the task at hand. While analysing errors may lead to conclusions \textit{what} does not work, \textit{why} it does not work is in many cases left to intuition. Other machine-leanring classes, like probabilistic or symbolic techniques, do not suffer from this problem, leading to an increasing interest in visualization techniques for neural networks. Most visualizations of sentence-representations to date focus on attention-based approaches, showing how words are aligned to each other, such as done by \cite{shen2018reinforced} or \cite{im2017distance}. To the best of our knowledge, no or little insights have been gained to understand the final real-valued sentence-representation in vector space. In this section we demonstrate, how a sentence-representation, arising from max-pooling, can be interpreted, using the Shortcut-Stacked Encoder\textsuperscript{$\dagger$} as the model to analyse. Intuitively, understanding how the Shortcut-Stacked Encoder\textsuperscript{$\dagger$} encodes information, can be helpful for the task at hand, of improving it using external resources. 
\newline

\noindent
While we did not manage to leverage the insights gained in this chapter to increase the performance, it might be helpful for future work.

\subsection{Insights on the sentence representation}\label{sec:insights_sent_repr}
In this section we show, how we analyse and interpret the information, that is present within the sentence-representions of the proposed model in Section \ref{sec:residual_encoder_def}. We do so by identifying, what kind of information is encoded, and demonstrate, that the sentence-representation can manually be adjusted in a meaningful way.
\subsubsection{Approach}
We use Shortcut-Stacked Encoder\textsuperscript{$\dagger$}, trained on \ac{SNLI}, for our analyses. This model creates for input each sentence $x$, consisting of natural language words, represented as $x_i$, a sentence-representation $r \in \mathbb{R}^{2048}$ with $r_j$ being the $j$th dimension of $r$. Arising from $x$, the representation $r$ captures the relevant information for the task at hand. Many applications represent natural text of variable length as a fixed real-valued vector without a deeper understanding what each $r_j$ actually encodes. We shed light into the dimension-wise meaning of the sentence-representation by identifying which word is responsible for the actual value of $r_j$. 
\paragraph*{Method}\label{sec:understanding1_method}
For simplicity, we explain the applied method on an example, using a more general neural architecture of \ac{LSTM}s, a simple uni-directional \ac{RNN}.
Figure \ref{fig:rnn} (left) shows the recursive workflow of such a \ac{RNN}, following the notations of \cite{goldberg2017Apr}.
\begin{figure}[tph!]
\centering
	\includegraphics[totalheight=5.5cm]{fig/rnn.png}
	\caption{General architecture of a \ac{RNN} (left). Example sentence in an unrolled \ac{RNN} (right).}
	\label{fig:rnn}
\end{figure}
Maintaining an internal state $s \in \mathbb{R}^m$, for $m$-dimensional representations, the network sequentially iterates over the input sequence $x$, aggregating in each timestep $i$ the previous state $s_{i-1} \in \mathbb{R}^m$ with the current input $x_i$, using the function $F$. This state $s_i$ is then used as input to the next iteration and additionally is output via a mapping function as $y_i \in \mathbb{R}^m$. Multiple implementation variants exist, specifying $F$ and what information is shared across time-steps. \ac{LSTM}s for instance use several neural gates, to learn what information should be used, should be output or forgotten. This procedure is clarified with an example sentence by unrolling the network in Figure \ref{fig:rnn} (right). In typical setups, a neural network may either choose to use $s_t$ or $y_t$ for a sequence length of $t$ as the final sentence representation \citep{goldberg2017Apr}, since the network iterated over the full input sequence and hence contains all relevant information, if optimized for it. Even though the \textit{architecture} of different versions of \ac{RNN}s is well understood and has a logical meaning, the actual procedure of deriving concrete representations within a trained model is hard to understand. We leverage the fact, that the Shortcurt Stacked Encoder uses max-pooling over all $y_i$ to gather the sentence representation, rather than using $y_t$ or $s_t$ by inverting this process. We do so, by identifying what $y_t$ has the highest value within a given dimension, and mapping this dimension to the word $x_t$ of the input sentence. As an example consider the sentence in Figure \ref{fig:rnn} (right). For each timestep $i$ a new vector $y_i$ is produced. As done by \cite{nie2017shortcut} we concatenate all $y_i$ to a matrix $\mathbb{R}^{m \times i}$, with $m$ being the representation size and each vector $y_i$ being the $i$th row-vector within $M$. Assuming a dimensionality of $m = 3$, an possible matrix $M$, as an example for the given sentence ``A child is swimming .'', is displayed in Figure \ref{fig:example_process_understanding}. 
\begin{figure}[tph!]
\centering
	\includegraphics[totalheight=2cm]{fig/example_process_understanding.png}
	\caption{Visualized example of extracting interpretable information of the max-pooled seentence representations with a dimensionality of 3.}
	\label{fig:example_process_understanding}
\end{figure}
Additionally to creating the sentence representation $r$ by applying row-wise max-pooling on $M$, we collect the vector $a$, containing the column indizes, that are responsible for the values within $r$. These can directly be mapped to the word of the source sentence, and hence be interpreted by humans. It should be noted, that due to the nature of the multi layer \ac{biLSTM} each $y_i$ does not only encode the word at $x_i$ but also its context. While this somehow may lead to less accurate mappings or noisy interpretations, we found that the chosen method is sufficient to gain some meaningful insights on sentence encoding.

\paragraph*{Analysed data}\label{sec:understanding1_analysed_data}
To reduce noise and for aiming for sentences, that Shortcut-Stacked Encoder\textsuperscript{$\dagger$} seems to have a proper understanging about, we sample 1000 sentence representations from the \ac{SNLI} train data in the following strategy: We group all sentence pairs ($p$, $h$), sharing the same $p$, and only keep groups, if all samples belonging to the same group are classified correctly. Thus, we reduce the amount of sentences, by removing all samples that are definetly not entirely correct understood by the model, which would be harder to interpret. For now, we are not interested in the actual relation between $p$ and $h$ and therefore create a pool of the remaining sentences, by treating $p$ and $h$ equally and splitting their connections apart. After removing duplicate sentences, the most frequent sentence length for the remaining representations is 8. To reduce noise that may arise from different sentence lengths, we only consider sentences of a length of 8 and randomly sample 1000 sentence representations. All experiments in this chapter are based on the same instances, unless otherwise stated.
\newline

\noindent
In addition to the representation values for each dimension, each sample contains the following information:
\begin{itemize}
\item \textbf{Words:} The words that triggered the maximum value for the representation.
\item \textbf{Word position:} Positional information about the responsible word within the sentence.
\item \textbf{Lemma:} The lemmata of the responsible words.
\item \textbf{\ac{POS}:} The \ac{POS} tags of the responsible words.
\item \textbf{Dependency Parse Tag:} The tags of the responsible tokens within the dependency parse tree.
\end{itemize}
Lemmatizing, \ac{POS}-Tagging and dependency parsing were conducted using spaCy\footnote{\href{https://spacy.io/}{https://spacy.io/}}.

\subsubsection{Detection of relevant dimensions}
As commonly done, when analysing data, we start by showing a quick overview for the sentence representations at hand. Typically, the \ac{SD} within a dimension (or any feature in general) correlates with its relevance for decision making. Naturally, a dimension, that does not change its value, and thus being close to constant, is not informative, while a value with a high \ac{SD} (or variation) can be considered informative \citep{Bishop2007}. We calculate \ac{SD} over all dimensions, depicted as a histogram in Figure \ref{fig:sd}.
\begin{figure}[tph!]
\centering
	\includegraphics[totalheight=8cm]{fig/sd.png}
	\caption{The standard deviation within a dimension of sentence representations (x-axis) by the amount of dimensions with the given standard deviation.}
	\label{fig:sd}
\end{figure}
We plot the standard deviations in a discrete space, using a bin size 0.05. For each of the 2048 dimensions we calculate its \ac{SD} to assign them to the correct bin. The amount of dimensions with the given \ac{SD} is shown on the y-axis. Note that the upper part of the plot is truncated for the sake of compactness. As can be seen, only a very tiny fraction of the dimensions shows a large variation, the vast majority contains more or less the same value, regardless of the sentence. This obviously does not mean, they contain no information at all, as they may only be used to encode information that is rarly present within the data (and not captured within the rather small subset of samples), however itserves as a reliable source, what dimensions are relevant to the model and which are not.

\paragraph*{A naive approach to identify dimensional encoding}
An intuitive approach to identify, what is encoded within the sentence representation, is, to find common similarities between the words across all sentences, that are responsible for the same dimension, neglecting the actual value, reached by each word. Especially, for the task of \ac{NLI}, we assume \textit{semantic}, \textit{syntactic} and \textit{positional} information to be required. Those can all be inferred using the features we extraced in Section §\ref{sec:understanding1_analysed_data}. Similarities between words heavily depend on the context they appear in \citep{dagan2000contextual}. For instance one could consider a car and an identical reconstruction in original size, but all made of plastic, as similar, whereas a horse is very distinct. Adding additional information, that one needs to reach a destination in short time, he or she is more likely to consider the horse similar to the car, desiding between those two options. This essentially comes to a major problem when investigating semantic similarities, without prior knowledge, of what attributes may be considered relevant. We therefore investigate the sentence representation, using excessive manual search, in a top down manner: We first search for patterns across many dimensions and many sentences in this section. In Section §\ref{sec:understanding2} we will look into some dimensions in detail.

\paragraph*{A tool for sentence representation visualization}
In order to evaluate many patterns with minimal time effort, we create a visualization tool, capable of dynamically generating any labelling scheme for responsible words, based on the features described previously. A sample visualitation is shown in Figure \ref{fig:find_position_1}.
\begin{figure}[tph!]
\centering
	\includegraphics[totalheight=4cm]{fig/finpone.png}
	\caption{An extraction of a grid-plot, showing dimensions with the position within the sentence of the word, responsible for the dimensional value.}
	\label{fig:find_position_1}
\end{figure}
This grid-plot visualizes in each row one particular dimension, listed on the left side as (\texttt{<rank in terms of \ac{SD}}\footnote{All dimensions are ranked by their \ac{SD}, giving an intuition of the expressiveness of the dimension.}\texttt{> - <dimension index>}). We color the responsible words for each dimension, based on the attributes of interest, shown on the left side. In this particular case, words are colored by their position within the sentence.  Each column refers to the same sentence along different dimensions. As a trade-off between explanatory power and clarity we always plot 300 sentences on 300 dimensions, which are eigther ordered by \ac{SD} or already pre-sorted by the frequency\footnote{Even though we only use 300 sentences and dimensions for plotting, calculatiions are based on all the selected data.} of a label of interest. In this particular case, dimensions are ordered by their frequency of words on position 1, meaning the dimension within the first row, received its values from the second word (index $1$) more than any other dimension. Looking for patterns across many sentences, we focus on horizontal lines with the same coloring, or colors referring to attributes that may be interpreted similarily. Vertical lines indicate differences across sentences with respect to the attributes of interest and their impact on the visualized dimensions.


\paragraph*{Interpretation of positional Information}
Word-ordering is crucial  with respect to the meaning of a sentence. We evaluate if certain dimensions are aligned to specific word positions, and hence only serve to encode the meaning of the word at a specific position within the sentence. Figure \ref{fig:find_position_1} shows dimensions, that are heavily influenced by the second word, indicated by the vast majority of green dots. And indeed, several, also very informative dimensions, are dominated by the second word (ignoring some noise, primarily stemming from other word positions from the beginning of the sentence). Considering the nature of sentences of \ac{SNLI}, as presented in Section §\ref{sec:snli}, this is however not enough evidence to conclude, those dimensions correspond (solely) to positional information within the sentence. Taking the merely simple sentence structures (or even only phrases) of \ac{SNLI} into account, it is very likely that the second word in most cases corresponds to a noun, presumeably describing the main aspect of the image. Optional preceding articles or adjectives may cause this noun to have varying positions between one and three. Looking at the coloring of vertical lines this assumption is backed up, as for each sentence, the responsible word arises fairly consitstently from the same position across most visualized dimensions, indicating, that this stems from the encoded attribute rather than noise. In general we find no meaningful\footnote{We do find several dimensions that mostly arise from the first word, yet the resulting value is mostly constant across all sentences and mainly stems from articles like ``The'' or ``A''.} dimensions encoding solely positional information, neither with absolute nor with relative positions, without being correlated strongly with another, more meaningful attribute.
\paragraph*{Finding syntactic dimensions}
This warrants more investigation using a different labeling scheme, and we look for clues based on \ac{POS} tags. Tokens are labelled using the Penn Treebank Part-of-Speech Tagset \citep{marcus1993building} and available in our data with the originally assigned labels. Figure \ref{fig:find_syntax} shows an extract of dimensions, labelled by \ac{POS} tags, pre-sorted, such that dimensions with any single dominant label are shown first. We aggregate different \ac{POS}-tags, refering to the same concept, together, thus for instance all nouns \texttt{NN}, \texttt{NNS}, \texttt{NNP} and \texttt{NNPS} are labelled as \texttt{NN}. Several patterns can be seen within this plot. Especially punctuation seems verly well presented at first sight (green and orange). Yet, looking into the actual data and considering their very low \ac{SD}, these dimensions seem less important. We observe similar issues for dimensions that are dominated by articles (orange). More interestingly are nouns (yellow), being very dominant in diverse dimensions, including dimension \texttt{1878} (fourth row), which is also well represented by the second word when checking for positional information (first row in Figure \ref{fig:find_position_1}). 
\begin{figure}[tph!]
\centering
	\includegraphics[totalheight=7cm]{fig/finsynpos3.png}
	\caption{An extraction of a grid-plot, showing syntatical information using the \ac{POS} tag with pre-sorted rows to have a single dominant label.}
	\label{fig:find_syntax}
\end{figure}
\noindent
This supports our intuition, that word positions are not directly encoded but merely correlated to other features, like an early noun in this case. Also verbs (blue) are well presented in two dimensions. This plot suggests, that indeed that model captures syntactic information to some extend, represented within the according dimensions respectively, however also shows some drawbacks of our initial naive approach:
\begin{enumerate}
\item \textbf{Correlation:} As we have shown, different attributes may correlate with each other, thus it is unclear, if a found pattern is a side product of a correlated feature, or the ``main thing'', being encoded by a dimension.  
\item \textbf{Non existent information:} Interpreting the meaning of a dimension by the responsible word is only possible if this word indeed characterizes the dimension. In case of positional information we can be certain, that each sentence contains a word that reflects the information with respect to our labeling scheme, as every sentence contains words with all positions. However when looking for encoded information that is not present amongs all sentences, another arbitrary word will still be responsible for the dimension's value. In our case, when finding syntactical clue for instance, most sentences have punctuation, nouns, determiners or verbs. A dimension representing adjectives however will always look very noisy, as it can only be represented by adjectives in the sentence contains such a \ac{POS}.
\item \textbf{Representation value:} We did not yet consider the actual real-valued representation, as used by the model for prediction. Especially considering the previous issue, we expect the model, to have some kind of encoding to differentiate, whether the information of a dimension is present or not.
\end{enumerate} 
While we will never completely get rid of the first issue, we try to remedy misinterpretations coming from all three issues by closely analysing dimensions separatley in Section §\ref{sec:understanding2} together with the actual values of $r$ in the dimension at hand. We reduce the shortcomings from the second issue by adding a filtering option, that we demonstarte in the next section, when identifying dimensions containing semantic information.

\paragraph*{Finding semtantic dimensions}
Looking at the actual data, we observe that several dimensions only include words referring to female humans. We investigate this finding by looking for dimensions that contain gender-specific information. Based on the data we create two wordlists for female\footnote{Words in the \textbf{female} wordlist: \textit{daughter, daughters, female, females, girl, girls, lady, mother, mothers, her, herself, she, sister, sisters, wife, woman, women}} and male\footnote{Words in the \textbf{male} wordlist: \textit{boy, boys, dude, father, grandfather, guy, guys, he, him, himself, his, husband, male, males, man, men, son, sons}} humans respectively. Following our observation, that more specific information causes more noise in the visualization, due to sentences not containing it, we only consider sentences containing at least one word from at least one wordlist. The resulting plot is depicted in Figure \ref{fig:find_male_female} with dimensions sorted by their informativeness according to \ac{SD}. Words are labelled as \textit{male} or \textit{female}, if they occur in the according wordlist, any other word than those is labelled \textit{OTHER}.
\begin{figure}[tph!]
\centering
	\includegraphics[totalheight=5cm]{fig/findmf.png}
	\caption{An extraction of a grid-plot, gender specific female using only sentences with words of pre-defined wordlists.}
	\label{fig:find_male_female}
\end{figure}
Several interesting findings are shown here. The first two dimensions, having the highest variation, do not distinguish between female or male humans, but instead jointly encode both information, seemingly focusing on humans with a specified gender. More importantly however, we observe several dimensions with only male (red) and OTHER (yellow) words responsible for its value, while others only arise from female (green) and OTHER only. Comparing these dimensions, we see in the grid plot, that some of them are strongly complementary to each other with respect to the encoded gender: We interpret dimensions that exclusively retrieve their values from female (or OTHER) words, to encode whether there is a female human being present in the sentence (labelled as female) or not (labelled as OTHER). Similarily, dimensions exlusively arising from male (or OTHER) words are likely, to encode whether a male human-being is within the sentence or not. Based on this interpretation, one can observe, that male-encoding dimensions are labelled as OTHER exactly for those sentences, that are labelled as female in female-encoding dimensions and vice versa. Note that all displayed dimensions contain the highest overall \ac{SD}, indicating that they are amongst the most expressive dimensions of the model. Being redundantly in several informative dimensions encoded, we conclude, that gender-specific information is highly relevant for \ac{SNLI}. This observation is in line with \cite{gururangan2018annotation}, who show that removing the gender information to create the hypothesis was a common heuristic, applied by the annotators when creating the dataset.

\subsubsection{Female and male dimensions}\label{sec:understanding2}
We rely on the method, described above, to manually find patterns, that are encoded in the sentence representations, and identify the corresponding dimensions. Following the drawbacks of our naive approach we conduct our dimension-wise analysis with respect to the actual values of the given dimension, that are retrieved from each word. We represent each dimension in a histogram by mapping the dimensional values from each sentence into a discrete space. In prelimitary attempts tried to automatically divide the one-dimensional feature space of a dimension into meaningful intervals, however this did not create meaningful representations. 
\paragraph*{Female and male dimensions}
Figure \ref{fig:mf_basic} shows a detailed view of two of the male-encoding dimensions with no filter applied, thus using all 1000 sentences. 
\begin{figure}[tph!]
\centering
	\includegraphics[totalheight=4.5cm]{fig/mf_basic.png}
	\caption{Representation visualitation with respect to genders of dimension \texttt{199} (left) and dimension \texttt{602} (right).}
	\label{fig:mf_basic}
\end{figure}
For each sentence we retrieve the value of the two displayed dimension and assign it into bins of size $0.05$, displayed on the x-axis. The amount of sentences containing a specific value is displayed on the y-axis for each bin, colored by the chosen labelling scheme, in this case, if they are within the chosen word-groups. Based on our previous observation we assume the gender-dimension to only encode whether, a human with the given gender is within the sentence or not. To distinguish terms for humans without a specified gender from male and female humans, we create  third category, containing words for humans without a specified gender \footnote{Words in the \textbf{gender-less} wordlist: \textit{parent, parents, friend, friends, person, people, familiy, student, adult, adults, couple, couples, child, children}}. Additionlly we move pronouns from our previous wordlists into new categories. 
Both dimensions undermine our initial assumption, that they encode whether a male human is present within the sentence or not. Clearly this is seperated by the value within the dimension. All high values arise from male words, most of them covered by our rather limited word-list. Not a single value stems from any word of the female word-list. While neutral words may be responsible for values within these dimensions, their influence is negligable. All words coming from lower-valued bins seem arbitrary, arising from the fact that some sentences do not contain any male words and thus a random word will take its place, resulting in a low value. Even though the distributions are different, both presented dimensions seem to encode the same information. For an even more detailed view, we focus on the individual words from the male wordlist in Figure \ref{fig:mf_detailed_m}.
\begin{figure}[tph!]
\centering
	\includegraphics[totalheight=4.5cm]{fig/mf_detailed.png}
	\caption{Detailed representation visualitation of different terms for human males of dimension \texttt{199} (left) and dimension \texttt{602} (right).}
	\label{fig:mf_detailed_m}
\end{figure}
We observe, that both dimensions encode a fine-grained differentiation between different words and their meanings, even within their high values. In both cases, ``boy'' scores a lower value than ``man'' as being \textit{less male}, indicating that this dimension not corresponds to the biological male gender but to attributes, that are generally assoziated with males. This only seems logical, as it is known that gender information is present in distributed word-representations \citep{mikolov2013linguistic}, that the model relies on. These again, are determined by their surrounding context, which obviously is dominated by male-\textit{assoziated} words. While both dimensions seem to encode the same information, based on the words reaching high values, the encoding of this information slightly differs. Dimension \texttt{199} has the tendency to score higher values if multiple males, namely ``boys'' and ``men'', are present, whereas dimension \texttt{602} reduces the value for plurals. 
\newline

\noindent
We simillarily investigate the female-encoding dimensions, depicted in Figure \ref{fig:mf_detailed_f}, already using the detailed labelling and observe the same principal encoding scheme. 
\begin{figure}[tph!]
\centering
	\includegraphics[totalheight=5cm]{fig/mf_detailed_f.png}
	\caption{Detailed representation visualitation of different terms for human females of dimension \texttt{845} (left) and dimension \texttt{311} (right).}
	\label{fig:mf_detailed_f}
\end{figure}
As for the male-encoding dimensions, all higher values within both dimensions arise almost exclusively from our female word-list. Younger females, namely ``girl[s]'', are encoded, using a lower value than ``woman'' or ``women''. Furthermore, the different encoding of both female dimensions of singular and plural is aligned with the differences, depicted in the male dimensions. Subsequent visualizations of other dimensions show similar results, such that higher valued words may easily grouped by some attributes, while low values words seem rather arbitrary. We conclude that each (or most) dimensions encode some specific information of any kind. We denote this information of a specific dimension as $\xi$ and use both formulations exchangable within the remainder of this thesis. High values within each dimension are used, if $\xi$ is present within the sentence, and low values, if $\xi$ is not present. Note that this explanation intuitively can be aligned with the max-pooling. Given that $\xi$ is within the sentence, the model adjusts its weights, such that the dimensions, encoding $\xi$, will result in high values, which naturally will be selected as being the highest amongst all values. In case $\xi$ is not within the sentence, any other arbitrary word will have the highest value of the specific dimension, however this will be significantly lower than in the previous case. We show further evidence for this explanation in the remainding subsections.

\paragraph*{Relevance of female and male dimensions}
\begin{wraptable}[12]{r}{7cm}
%\begin{table}[]
\centering
\begin{tabular}{llll}
\textbf{$|r|$} & \begin{tabular}[c]{@{}l@{}}\textbf{MLP}\\ \textbf{size}\end{tabular} & \textbf{Acc. (train)} & \textbf{Acc. (dev)} \\
\toprule
2 & 6& 42.26\% & 42.87\%\\
4 & 12& 47.11\%& 47.41\%\\
6 & 18& 47.50\%& 48.64\%\\
8 & 24& 49.54\%& 50.37\%\\
\bottomrule      
\end{tabular}
\caption{Accuracies achieved on \ac{SNLI} using $|r|$-dimensional sentence representations of gender-specific dimensions.}
\label{tab:relevance_mf_nn}
%\end{table}
\end{wraptable}
We now closer examine, how relevant those identified dimensions actually are, when the predicting the relations on \ac{SNLI}. We conduct an experiment with sentence representations, solely consisting of dimensions that we identified to encode gender-specific information. Specifically we found four dimensions to encode each gender respectively. In the first experiment, we only consider subsets of these dimensions as sentence representations, and train a new neural network for each subset, consisting of an equal amount of female and male dimensions \footnote{We select those dimensions of all sentence-representations from the fully trained Shortcut-Stacked Encoder\textsuperscript{$\dagger$}}. We train the model for 5 iterations using the same hyperparamters as in the Shortcut-Stacked Encoder\textsuperscript{$\dagger$}. Solely the size of the hidden layer is changed with respect to the size of the sentence representation, due to being tremendously reduced (by only considering male and female dimensions). The results in Table \ref{tab:relevance_mf_nn} show, that a sentence representation, solely consisting of one dimension per gender, reaches an improvement of about 9 points in accuracy over a random baseline with 33.33\%. Adding more dimensions somehow reduced the additional improvements, indicating some redundancy, but also some distinct information is encoded within those dimensions. About half of the data can be classified correctly based on only 8 dimensional sentence-representations, encoding the gender-information of the sentence. This indicates, that these dimensions are highly relevant within the model, however all these observations may also arise from the newly trained model, learning patterns, that are not considered by the original model. We thus conduct another experiment to shed more light into the impact of the detected dimensions for the Shortcut-Stacked Encoder\textsuperscript{$\dagger$}.

\paragraph*{Inverting gender information in the sentence-representation}
Knowing, what information is encoded and how this is done, we try to twist the sentence-representations before they are fed into the classifying \ac{MLP}, hence identify, if it is possible to exploit the gained knowledge for adjusting the represented meaning, without adapting the actual input sentences. Specifically, we try to invert the meaning of the found gender-specific dimensions: If the sentence-representation originally contains information that a male or female human is present, we change it to be not present and vice versa. Let $d^i_j$ denote the value of the $i$th dimension within the $j$th sentence representation. For all $i$, referring to gender-encoding dimensions,\footnote{\textbf{Male} dimension indizes: 89, 199, 280, 602; \textbf{Female} dimension indizes: 311, 609, 845, 1730} we calculate the maximum reached value, denoted as $d^i_{max}$, and minimum reached value, denoted as $d^i_{min}$, over all $n$ sentences, whereas $n$ is the amount of sentences in \ac{SNLI} train data.
\begin{equation}
d^i_{max} = \arg\max_{v^i}(v^i | v^i \in \{d^i_0, d^i_1, \cdots,d^i_{n-1}, d^i_n\})
\end{equation}
\begin{equation}
d^i_{min} = \arg\min_{v^i}(v^i | v^i \in \{d^i_0, d^i_1, \cdots,d^i_{n-1}, d^i_n\})
\end{equation}
We further calculate the new value $\bar{d}^i_j$, replacing the original value $d^i_j$, for all relevant $i$ using the following equation. Note that we replace $d^i_j$ by $\bar{d}^i_j$ prior to the feature concatenation, ensuring to overwrite all relevant features. Basically we mirror each value on the dimension's mean, ensuring that the resulting values are within the valid range for the given dimension.
\begin{equation}\label{eq:invert}
\bar{d}^i_j = \frac{d^i_{max} + d^i_{min}}{2} + \left(\frac{d^i_{max} + d^i_{min}}{2} - d^i_j\right) = d^i_{max} + d^i_{min} - d^i_j
\end{equation}
Rather than using the sample mean from all sentences, which would be heavily influenced by how much the encoded information is represented within the data, we calculate the mean based on the outer values, intending to focus on the information-encoding aspect. Considering the distributions of the dimensions, having two peaks, either low- or high-valued, we assume this method to be appropriate for our experiment.

\paragraph*{Evaluation of inverted gender-dimensions}
We use the proposed method for the full \ac{SNLI} train and dev data and report our results in Table \ref{tab:inverted_mf_results_acc}, together with the original performance of the used Shortcut-Stacked Encoder\textsuperscript{$\dagger$}. 
\begin{table}[tph!]
\centering
\begin{tabular}{cccccc}
\textbf{Inverted dimensions} & \textbf{Inverted sentences}  & \textbf{Acc. (train)} & \textbf{Acc. (train) +/-} & \textbf{Acc. (dev)} & \textbf{Acc (dev) +/-} \\
\toprule
None                   & None                   & 87.41        & 0.0              & 85.31      & 0.0           \\
\midrule
1 female, 1 male    & premise, hypothesis & 87.36        & -0.05            & 85.25      & -0.06         \\
2 female, 2 male    & premise, hypothesis & 87.25        & -0.16            & 85.19      & -0.12         \\
3 female, 3 male    & premise, hypothesis & 87.08        & -0.33            & 84.97      & -0.34         \\
4 female, 4 male    & premise, hypothesis & 86.82        & \textbf{-0.59}            & 84.78      & \textbf{-0.53}         \\
\midrule
1 female, 1 male    & hypothesis          & 87.05        & -0.36            & 84.73      & -0.58         \\
2 female, 2 male    & hypothesis          & 84.76        & -2.65            & 82.29      & -3.02         \\
3 female, 3 male    & hypothesis          & 80.23        & -7.18            & 78.28      & -7.03         \\
4 female, 4 male    & hypothesis           & 73.20        & \textbf{-14.21}           & 71.69      & \textbf{-13.62}    \\
\bottomrule    
\end{tabular}
\caption{Results in terms of accuracy of inverted gender-specific dimensions on \ac{SNLI} train and dev set.}
\label{tab:inverted_mf_results_acc}
\end{table}
The table shows a range of experiments with an increasing number of dimensions being inverted, on either both sentences or only on the hypothesis. It can be seen, that even after inverting all four dimensions for each gender, the performance only slightly drops, when applied on both sentenes. This indicates that the performed equation (\ref{eq:invert}) indeed is sufficient to invert the encoded information to a high degree, since inverting the gender in $p$ and $h$ simultaneously should not have a large impact on the final prediction\footnote{We select only 8 out of 2048 dimensions, that are highly representative for the gender-specific meaning. Correlated information however also has an impact on other dimensions, that we left untouched. Thus we don't claim to have inverted the full gender-specific meaning, but a crucial amount of it.}. More importantly, when only the hypothesis' representation is changed, we observe, that inverting a single dimension reduces the model's performance only slightly. Increasing amount of inverted dimensions, also increases the negative impact on the overall accuracy. We conclude that this is due to redundant information, most likely coming from dropout. 
\paragraph*{Analysis of inverted gender-dimensions}
In order to actually see, that sentence-meanings shifted, according to our expectations, namely male individuals should be interpreted as female and vice versa, we analyse the predictions of the data. The results for our observations, when looking for differences between the models prediction using the untouched or inverted (using all eight dimensions, inverting the hypothesis only) sentence-representations, are shown in Table \ref{tab:inverted_mf_results_samples}. The upper part of the table depicts samples with human main actors having a specified gender, that are correctly classified by the original model. By inverting the mentioned dimensions, we invert the gender-aspect of these actors and subsequently a ``woman'' in the hypothesis is afterwards encoded similarily to a ``man'' by the model within the sentence. We observe that the vast majority of samples, containing the same gender in premise and hypothesis, flip the predicted label when being inverted, hence following this interpretation. Some samples without explicit gender information, like ``dog'' in the lower part of the table, remain with the same label. 
\begin{table}[tph!]
\resizebox{\textwidth}{!}{%
\centering
\begin{tabular}{llcc}
\textbf{Premise} & \textbf{Hypothesis}  & \specialcell{\textbf{Prediction}\\\textbf{(original)}} & \specialcell{\textbf{Prediction}\\\textbf{(inverted)}} \\
\toprule
% w gender good contradiction
A blurry \textit{woman} eating fish. & The \textit{woman} is eating dinner. & neutral & contradiction \\
A \textit{woman} practicing for tennis. & A \textit{woman} practices tennis & entailment & contradiction \\
Three \textit{men} sitting behind a building. & Three \textit{men} are sitting. & entailment & contradiction \\
% w gender good entailment
A \textit{male} in a green jacket points an imaginary shotgun at the sky. & A \textit{woman} in a green jacket pointing an imaginary gun at the sky.& contradiction & entailment \\
A young \textit{woman} with a ponytail climbs a white stone structure. & A young \textit{man} has a ponytail. & contradiction & entailment \\
A little \textit{girl} in brown is playing with two hula-hoops. & The person playing with hula-hoops is \textit{male}. & contradiction & entailment \\
\midrule
% w/o gender, good
Two dogs standing in the snow. & The dogs are looking in the same direction. & neutral & neutral \\
Two people dancing outside. & Two people dancing. & entailment & contradiction \\
A country band is playing. & A group is playing music. & entailment & contradiction \\
% w/o gender, bad

\bottomrule    
\end{tabular}}
\caption{Comparison of samples between their predictions based on the original and gender-inverted sentence representations.}
\label{tab:inverted_mf_results_samples}
\end{table}
Especially for human main actors without a specific gender, we observe unexpected predictions. One possible explanation could be, that since words like ``band'' or ``people'' have no specified gender, the inversion of the hypothesis suggests that people of both gender are present in the sentence. Unlike ``dog'', those terms are encoded similarily to humans with a specified gender, since both refer to human actors, which yields in according information in other dimensions. Hence, an example regarding a ``dog'' is less likely to have unwanted side-effects, as the gender presumeably is only considered by the model in conjunction with other dimensions, representing human-beings.
\newline

\noindent
While this experiment intentionally does not improve the accuracy on \ac{SNLI}, it shows that it is possible to adjust the sentence representation in a meaningful manner, using the gained insights on how information is encoded. The results give evidence, that in the majority of cases, the new meaning corresponds to the initial intention, yet also comes with some minor side-effects, undermining the need to have a deeper understanding how the representations are in fact used by the classifier. 
\subsubsection{Other semantic dimensions}
Following the idea of high valued words being representative for $\xi$, encoded by a dimension, we analyse more than 100 additional dimensions, finding mostly semantic similarities between the words of interest (high values). Below, we give an overview about the semantic aspects, covered in the representations, and provide sample words, taken from a single dimension each time:
\begin{itemize}
\item \textbf{Mixed:} The vast majority of dimensions encode even within higher dimensions several different ideas, that can be grasped by looking at the words. For instance one dimension simultaneously considers words as relevant that are related to fast movement or lonely emotion assotiations simlutaneaously (running, runs, race, jogging, no, alone, dry, empty, crying, timid). Another dimension assign all very high values to nouns that are possible arguments to the word \textit{play}, namely instruments and sports, somewhat lower but still very high words reflect assoziated verbs or tools for sporty or artistic activities (soccer, football, baseball, drums, tennis, accordion, guitar, saxophone, dancing, swimmig, sining, painting, boat, bicycle, surfboard). In fact, most dimensions contain words within their high values that may easily be clustered in several groups. Sometimes a single dominant common meaning exists. All examples shown below of course include other words, that may be grouped as an additional category and not necessarily are directly related to the assumed encoded information. However if there is a highly dominant pattern, we ignore words that are divergent to this meaning, considering it as noise. Given the fact, that the representations actually consider the context around the responsible word, we are only able to get an impression of the meaning rather than a accurate defintion anyway, by solely looking at the individual words. Yet, dimensions with several different relations can also indicate, that some dimensions are not individually responsible for a specific $\xi$. In conjunction with another dimension, those meanings of the words may be seperated from each other, yet we did not inviestigate further into this direction.
\item \textbf{Community:} Several dimenions encode different communal aspects, like family related topics (children sibling, moter, wife, school) or sozial events (friends, championship, lunch, family, baseball, party)
\item \textbf{Children:} Different dimension represent children in varying contexts. These dimensions show, that indeed context is captured within the dimensions and they not solely rely on the word, we investigate. Yet, they can be interpreted. Specifically we find several dimensions with children in playful contexts (boy, girl, young, playing, game, teens, skateboarding, soccer) or in the context of caring and comfortness (boy, girl, young, child, sleeping, hungry, tiny, napping, sad, asleep).
\item \textbf{Locations:} A huge amount of dimension encodes locational information. This may be for instance city or building related (pool, inside, restaurant, sidewalk, floor, bar, classroom, museum, downtown, building) water oriented (beach, pool, water, lake, river, mud, ocean) or referring to different grounds (street, beach, road, sidewak, grass). Several dimensions show topic related locations, together with possible activities, and are harder to categorize (street, beach, outside, park, soccer, rock, truck, boat).
\item \textbf{General atmosphere} Some dimensions consist of a broad range of words, however it still is obvious, that there is a higher common meaning.  One dimension for instance ranges from activities to locations or food, all seemingly indicate some level of ``lazy comfortableness'' (sitting, inside, sleeping, bed, room, dinner, cream, milkshakes, doll).
\item \textbf{Activities} Several dimensions include some kind of activity related words, consisting of both, verbs and nouns, clearly showing common attributes (ball, game, race, competition, skateboard, concert, artist), or encoding verbs that usually take positional arguments (walking, sitting, running, standing, walks, jumping), while others solely focus on only one of these meanings (standing, stand, stands, feet).
\item \textbf{Others:} We found more dimensions, not fitting into a broader category, but still encoding very specific information. For instant one dimension considers everything that has to to with ``wearing clothes'' as a high value (wearing, dressed, covered, shirt, umbrellas, naked, jersey, dress). Another dimension clearly consists of terms, describing humans using their profession or activity (player, vendor, skier, musicians, clown, workers, jockeys, artist).
\end{itemize}

\noindent
While it usually is hard to specifically name the attribute, that most of the high valued words within a dimensions have in common, it is usually very straightforward to grasp the general idea. All these words give valuable information, that enables us to interpret sentence-representations, a large advantage considering that initially nothing was known. Almost exclusively we found these common ideas to be based on the semantic level.

\subsubsection{Syntactic dimensions}
After extensively looking for semantic information in the sentence-representation, we investigate how much syntax is encoded by looking looking for \ac{POS} and dependency parse tags.

\paragraph*{Verbs and adjectives}
We identify syntactic patterns across the sentences, looking for verbs and adjectives using \ac{POS} tags. One dimension, that is highly dominated by verbs, is depicted in Figure \ref{fig:find_syntax_vb} (left).
\begin{figure}[tph!]
\centering
	\includegraphics[totalheight=6cm]{fig/find_syntax_vb.png}
	\caption{Dimension \texttt{713} encoding verbs (left) and dimension \texttt{2020} encoding adjectives.}
	\label{fig:find_syntax_vb}
\end{figure}
Looking at the actual words, that are responsible for the high values within the dimension however, we find that there is also a semantic commonality between them, with verbs mostly being \textit{playing, walking, sitting, running, standing, swimming}. Several interpretations on what they have in common are plausible, like all taking locations or places as arguments\footnote{Especilly considering that there is a large amount of other verbs present within \ac{SNLI}.} or all being some kind of physical activity\footnote{Having different scales how physically intensive it is, in the sense of being sportif.}. Nouns, scoring high values, exclusively denote sport types like \textit{football, basketball, tennis, baseball, volleyball, hockey}, which represent in combination with a verb also a physical activity. Looking at the words of the adjective-encoding dimension, depicted in Figure \ref{fig:find_syntax_vb} (right), we observe an even more obvious semantic relation, since all adjectives\footnote{High valued words of the \textbf{adjective} dimension: young, little, old, small, older, fat, large, elderly, lean, middle-aged, ...} are typically used to describe people, even though many different adjectives do exist in \ac{SNLI}. We face the same problem as described earlier, that different attributes correlate (in this case semantic and syntactic attributes), making a definite interpretation hard. We also observe, that this dimension, like some others, differs strongly with the gender-specific dimensions in their distributions. Gender-specific dimensions consist of two clear peaks, intuitively because they basically encode two states: Either the gender is present or not. This dimension however is encoded using a relatively large range of values, all being relatively equally represented. Event though we do not go deeper into a fine-grained analysis of dimensional values, due to the impact arising from encoded contexts, we assume that other information, as in this case, can be scaled. 
\paragraph*{Subjects and objects}
The differentiation between subject and object, that can be identified using dependency parsing, seems highly useful for classifying image captions. While the subject most likely refers to the main object, depicted in an image, the object may serve as a more informative explanatio, however most likely being less relevant. To identify whether the model learns equivalent information, we look for subjects and objects respectively. We also look for predicates, however this is especially noisy, since many sentences in \ac{SNLI} actually are noun phrases and thus lack a main verb. The closest, to encoding this information, even by using dependency parsing labels, was the dimension in Figure \ref{fig:find_syntax_vb}. We find the two dimensions in Figure \ref{fig:find_syntax_subj_obj} for both, subject and object, respectively.
\begin{figure}[tph!]
\centering
	\includegraphics[totalheight=5cm]{fig/find_syntax_subj_obj.png}
	\caption{Dimension \texttt{757}, encoding the subjects (left), and dimension \texttt{1840} encoding objects (right) of sentences.}
	\label{fig:find_syntax_subj_obj}
\end{figure}
Similarily, as when analysing the verbal dimension, the first sight suggests that indeed information, as from a parse tree, is encoded within those dimensions. While this actiually may be true, undeniably both dimensions retrieve high values from words, that are also semantically highly related. Those of the dimension, seemingly encoding subjects (left), exlucively consist of words referring to people\footnote{High valued words of the \textbf{subject} dimension: man, woman, girl, boy, men, women, boys, girls}. While they all are very likely to encode a very importnat aspect and also are the subject of the sentence, other subjects, refering to anything else than humans, are not considerd by this dimension. Hence, it seems more likely that the semantic relationship is encoded and just happens to correlate the subject. The object-encoding dimension (right) shows the same phenomen, solely encoding words referring to places\footnote{High valued words of the \textbf{object} dimension: street, beach, pool, outside, park, road, restaurant, sidewalk, grass, city, ...} with a high value.
\newline

\noindent
We clearly see that syntactic information is indeed encoded, both for the dependency parse information as well as \ac{POS}. Yet, those dimensions highly correlate on the semantic level and it seems much more likely that the identification of these semantic patterns is sufficient for the model, to rely on it, encoding syntactic information. This obviously is coming from \ac{SNLI}, with the majority of sentences regarding people. Another plausible interpretation is that the model does not actually require any syntactical knowledge for the task, based on the simple sentence structure. We conclude that whether the model indirectly uses syntactic information, originating from semantic features, or solely leverages semantic information, not relying on syntax at all, is matter of the perspective, the truth lies probably somewhere in between. 

\subsection{Insights on the sentence alignment}\label{sec:insights_sent_alignment}
We have shown, that dimensions represent a specific information $\xi$ of any kind. High values within these dimensions indicate, that this $\xi$, is present while low values indicate it is not present in the sentence. In this section we analyse, using the newly gained insights, how the models finally aligns the encoded information in the sentence-represenations to predict the entailment relation label. For our analysis we sample 150 premises, each with one hypothesis for each label respectively, that are all classified correctly by Shortcut-Stacked Encoder\textsuperscript{$\dagger$}. 

\subsubsection{Alignment analysis on a single sample}
We first analyse single samples in order to identify plausible strategies for the network, when mapping both sentence-representations, initially knowing only very little, how the network actually leverages from the information. We demonstrate our results with the samples premise:
\begin{center}
\begin{tabular}{rl}
	\textbf{Premise:} & A woman sitting in the dirt.
\end{tabular}
\end{center}
and its three hypothesis, one for each label:
\begin{center}
\begin{tabular}{rll}
	\textbf{Entailment:} & There is a woman sitting \textit{outside}. \\
	\textbf{Neutral:} & A \textit{dirty} woman sitting in the dirt. \\
	\textbf{Contradiction:} & A woman \textit{standing} in the sand. \\
\end{tabular}
\end{center}
We highlight the words within each hypotheis, that we consider relevant for the correct label, based on our human judgement. The entailing hypothesis describes, for the most part, the same setting as the premise. Since ``dirt'' usually appears ``outside''\footnote{Based on its context in conjuntion with ``sitting'', ``dirt'' is most likely used in the sense of being a dirty outside ground.}, we consider this change of words as a generalitazion, meaning ``outside'' includes ``dirt'' (amongst others) in this context. The neutral hypothesis introduces new information about the woman being ``dirty'', which is very plausible, yet not explicitely given in the premise. The contradicting hypothesis clearly is incompatible with the premise, as the woman can either be ``standing'' or ``sitting'', but not both. In this subsection we show, that indeed, the identified differences can be easily observed when looking at both sentence-representations simultaneously, for the entailing and contradicting hypothesis. For the sake  of brevity we omit the analysis of the neutral sample in this subsection, as it does not give any additional insights.

\paragraph*{Visualizing the entailment relation}
We start by visualizing the relation between the premise and the entailing hypothesis. We assume that the model will rely on the the information $\xi$ encoded within a given dimension to compare the meanings of two sentences and infer its relation. 
\begin{figure}[tph!]
\centering
	\includegraphics[totalheight=7cm]{fig/alignment_entailment_sample_general.png}
	\caption{Word alignments of an entailing sentence pair either by counting all shared dimensions (left) or only dimensions with at least a value of 0.2 (right).}
	\label{fig:alignment_entailment_sample_general}
\end{figure}
Figure \ref{fig:alignment_entailment_sample_general} (left) shows the alignment between between the premise (y-axis) and the hypothesis (x-axis) by counting all dimensions $d_i$, wheras $i$ is defined by enumerating all dimensions, arising from each word. For each word, we identify all dimensions $d_i$ that are represented by the word in the respresentation. The intersection of a word from the premise and a word of the hypothesis is the total amount of all dimension with the same identifier $i$, that both words represent. Thus, for instance, ``dirt'' and ``outside'' have 176 dimensions in common. This plot is quite noisy, since it does not differentiate between different values, thus even dimensions that encode information which is not given in both sentences are arbitrarily aligned between two words. Following our insights from the previous section, we filter out all dimensions that do not at least have a value of 0.2 in both sentence representations in Figure \ref{fig:alignment_entailment_sample_general} (right), presumeably resulting in only compared $\xi$ that is present in both sentences. It can be observed that by applying this filtering the main aspects of both sentences (``woman/woman'', ``sitting/sitting'', ``dirt/outside'') are stongly aligned with each other, while the remaining words only show little similarities w.r.t. their encoding. This incidates, that aligning both sentences intuitively can result in the entailment label within the examined example.
\newline

\noindent
Note that the actual Shortcut-Stacked Encoder does not only rely on the original sentence representations only, but also combines them using element-wise multiplication and difference. 
\begin{figure}[tph!]
\centering
	\includegraphics[totalheight=7cm]{fig/alignment_entailment_sample_mult.png}
	\caption{Visualitation of an entailing sample with applied element-wise multiplication either using the mean (left) or maximum (right) product of all shared dimensions for each word pair.}
	\label{fig:alignment_entailment_sample_mult}
\end{figure}
While element-wise difference intuitively serves a direct comparison of the encoded information per dimension, the effect of the multiplication feature is less obvious. Figure \ref{fig:alignment_entailment_sample_mult} (left) visualizes the mean product of both sentence representations using element-wise multiplication as described in Section §\ref{sec:residual_encoder_def}, averaged by the amount of shared dimensions as counted in Figure \ref{fig:alignment_entailment_sample_general} (left). One can observe that the plot, arising from element-wise multiplications, similarily highllights the same relevant word relations as seen in the previous plot. Showing that similar information is present in both sentences, this indicates that it serves as some kind of soft \texttt{AND}-operator. As this plot again might be heavily influenced by irrelevant relations we also visualize the \textit{highest} elementwise product, as opposed to the mean, of all shared dimensions between two words in Figure \ref{fig:alignment_entailment_sample_mult} (right), showing a similar pattern.

\paragraph*{Visualizing the contradicing relation}
We visualize the shared dimensions of both contradicting sentences in Figure \ref{fig:alignment_contr_sample_general} (left) in the same manner, as done for the entailment sample, by only counting dimensions that achieve a value of at least 0.2 for both words of each word-pair. 
\begin{figure}[tph!]
\centering
	\includegraphics[totalheight=7cm]{fig/alignment_contr_sample_general.png}
	\caption{Visualitation of a contradicting sample by counting meaningful shared dimensions (left) and meaningful distinct dimensions 8right) amongst pairs of words.}
	\label{fig:alignment_contr_sample_general}
\end{figure}
Note, that not only the entailing word-pairs show similar values within several dimensions, but also ``sitting'' and ``standing'' share the same meaning in some cases. This obviously makes sense, as both verbs are similar w.r.t. several aspects. Since the Shortcut-Stacked Encoder creates sentence representations without looking at the other sentence (without inter-sentence attention) it must encode \textit{all} information, that might be relevant, and is not able to focus on specific relations, that would be crucial for this particular sentence pair. Thus, we also count dimensions, that are distinct between two word pairs, depicted in Figure \ref{fig:alignment_contr_sample_general} (right). This is the case for the majority of cases, especially since commpletely unrelated words are encoded by a high amount of distinct dimensions. In order to remove noise, coming from this issue, we apply two thresholds in our visualization:
\begin{itemize}
\item \textbf{Ensure meaningful relations:} To exclude the counts of dimension between unrelated words, we only consider word-pairs with at least 5 meaningful (in the sense of both values reaching at least 0.2) dimensions. This is motivated by the assumtion, that the model needs to learn which dimensions can be aligned in a meaningful way, which only is plausible if both words also share at least some commonalities. 
\item \textbf{Ensure meaningful value:} Only considering word-pairs with meaningful relations, we only count dimensions that are distinct for both words if the word of interest reaches at least a value of 0.2, thus encodes the presence of $\xi$.
\end{itemize}
We observe, that indeed ``sitting'' and ``standing'' are encoded using a lot more different dimensions than shared dimensions, and take a closer look at those in Figure \ref{fig:contradiction_alignment_unshared_dimwise}. 
\begin{figure}[tph!]
\centering
	\includegraphics[totalheight=7cm]{fig/contradiction_alignment_unshared_dimwise.png}
	\caption{Dimension-wise visualitation of distinct information represented by \textit{sitting} in the premise and \textit{standing} in the hypothesis.}
	\label{fig:contradiction_alignment_unshared_dimwise}
\end{figure}
The plot shows all ``meaningful'' dimensions, that only encode $\xi$ coming from \textit{one} of both words. Each dimension show two bars, the left bar indicates the value within the representation of $p$, the right bar of $h$. Colors give information about the word, which is responsible for the value. Additionally we leverage from the labelled dimensions, gained for some dimensions in the previous section, by providing sample words of the general meaning encoded by eaach dimension. We only show these indicators for dimensions that we labelled prior to the visualitation, to not be biased towards a specific interpretation. We observe that the identified dimensions indeed highly differ in their value serving as possible features to detect contradiction. This plot also is in line with previous conclusions, that not-given $\xi$ results in low values, coming from arbitrary words. We also see, that knowing $\xi$ (as provided by the labels), is helpful for understanding the shown values, since the labelled dimensions correspond to the responsible word, if having a high value.
\subsubsection{Approach for a general alignment understanding}\label{sec:approach_general_alignment_understanding}
Our previous results show that it theoretically is possible for the model, to align relevant dimensions an thus infer the entailment label. Doing so, it could differentiate between contradicting or entailing meanings of two sentences. Other than the fact, that the model predicted the examined sentence correctly however, our findings are based on a single sample and show how the model \textit{could}, not actually \textit{does} leverage these information. To also take into account the actual prediction based on the \ac{MLP}, we conduct another experiment. Again following our conclusions that high values indicate the presence of $\xi$, we formulate a very simple form of lexical entailment w.r.t. to our identified encoding schemes. Let $I_p$ and $I_h$ be the sets of all $\xi$, that are present within $p$ and $h$ respectively. We further assume that lower values within a dimension generally represent less information w.r.t. $\xi$, the information encoded by the dimension:
\begin{enumerate}
\item The hypothesis contains a subset of information ($I_h \subseteq I_p$) would either result in paraphrasing ($I_h \equiv I_p$) or in less specific information in $I_h$, consequently being more general. We expect both cases to be labelled as entailment. We assume that for instance a hyponym ``monkey'' of the hypernym ``animal'' contains the same high dimensions as its hypernym and additionally more information that is specific for being a ``monkey''.
\item The hypothesis contains a true superset of information ($I_p \subset I_h$) results in the opposite case of the one above. The additional information $I_h \setminus I_p$ is possibly true based on the premise, yet not given. We expect this case to be labelled as neutral.
\item The hypothesis and premise contain different information ($I_p \nsubseteq I_h \land I_h \nsubseteq I_p$). We expect the model to predict contradiction if the amount of exclusive information in $I_p$ and $I_h$ is relatively large.
\end{enumerate}
In the following, we will refer to those assumpion using their numbering (1), (2) or (3) respectively. First however, we demonstrate this intuition based on an artifically created sample:
\begin{center}
\begin{tabular}{rl}
\textbf{Premise:} & A \textit{green} man is running on the street.
\\
\textbf{Hypothesis:} & A man is running on the street.
\end{tabular}
\end{center}
This is predicted as entailment, as expected by assumption (1) by our model. After swapping the premise with the hypothesis, the predicted label is neutral, as expected by assumption (2).
\begin{figure}[tph!]
\centering
	\includegraphics[totalheight=6cm]{fig/sample_coverage_prob.png}
	\caption{Visualitation of a sample sentence pair with explanatory guides for interpretation.}
	\label{fig:sample_coverage_prob}
\end{figure}
We visualize both sentence-representations in Figure \ref{fig:sample_coverage_prob} and include additional hints to explain how this visualzation can be read, as we use will use the same technique when looking at multiple samples simultaneously. Intending the figure to serve the validation of our claims, we create it in the following manner: Let $D = \{i \in \mathbb{N} | 1 \leq i \leq n\}$ denote the set of all $n$ dimensions. Furthermore let $p_i$ and $h_i$ denote the value within the $i$th dimension within $p$ and $h$ respectively. We divide the value range of all values within $\{p_i | i \in D\}$ and $\{h_i | i \in D\}$ into a discrete space, using bins of size 0.1, displayed at the y-axis for $p$ and the x-axis for $h$ with their lower bounds. For each $i \in D$ we identify the corresponding bins, based on the values $p_i$ and $h_i$ and increment the intersecting field by one. Thus, for instance, 1171 dimensions have a value $-0.0241 \leq p_i < 0.0759$ for the premise, while also having a value $-0.0244 \leq h_i < 0.0756$ within the hypothesis. Following our insights, that low valued dimension encode the absence of $\xi$, we consider the uper left corner as irrelevant for the classification of the relation of both sentences. Arising from the same observations, the diagonale, marked by the red recangle, corresponds to $\xi$ that is present in both, $p$ and $h$. Subsequently, everything that is above this diagonal represents $\xi$, that only is present within $h$ and accordingly, everything to the left of the diagonale is only present within $p$. We investigate the origin of all $p_i$ to the left of the diagonal and observe, that they exclusively emerge from the word ``green'', which is the only additional information, given in $p$.
Knowing that this naive assumption does not completely hold, we evaluate it on the chosen 450 correctly classified examples. While we find evidence for (1) and (2), we will show why (3) is not sufficient. 

\subsubsection{Entailment analysis}
We conduct the same experiment, as conducted using a single sample in Section §\ref{sec:approach_general_alignment_understanding}, over 150 correctly classified samples with the gold label \textit{entailment}. The resulting plot in Figure \ref{fig:entailment_uninversed} is calclualted identically to the plot with thee sample sentences, however displaying the mean amount over all sentence-pairs, rather than the the absolute amount. We observe that indeed, the majority of sentence-pairs contains more information within $p$ than in $h$, undermining our  assumption (1). On the other hand only very few information are present in $h$ and not in $p$. Yet, in order to get a better understanding, we seperate entailment relations based on paraphrasing from entailment based on generalitzation, by re-predicting the sentence-pairs ($p$, $h$) with premise and hypothesis swapped, as ($h$, $p$). 
\begin{figure}[tph!]	\centering
\includegraphics[totalheight=6.5cm]{fig/entailment_uninversed.png}
	\caption{Visualization of 150 sentence pairs ($p$, $h$), correctly labelled as entailment.}
	\label{fig:entailment_uninversed}
\end{figure}
We expect all sentence-pairs, that are predicted \textit{entailment} for ($p$, $h$) and ($h$, $p$) to be paraphrasing. Accordingly, we consider all entailing sentence-pairs, that are re-predicted as \textit{neutral} after swapping, to arise from generalitation, with the more general sentence $h$ now being the premise. The resulting label distribution after swapping, based on the prediction of the Shortcut-Stacked Encoder\textsuperscript{$\dagger$}, is listed below:
\begin{itemize}
\item \textbf{Entailment:} 11 samples (7.3\%)
\item \textbf{Neutral:} 111 samples (74.0\%)
\item \textbf{Contradiction:} 28 samples (17.7\%)
\end{itemize}
Based on the model's prediction, the majority of cases are described by the second scenario. Only very few samples show the same encoding in both sentences, resulting in entailment. As the re-prediction to contradiction is rather counter-intuitive, we take a look at the actual data. As it turns out, in addition to obvious misclassifications, many of the samples contain quite specific $p$ with a highly general $p$, for example (before swapping):
\begin{center}
\begin{tabular}{rl}
\textbf{Premise:} & A girl reaching down into the water while standing at the edge of a river. \\
\textbf{Hypothesis:} & The girl is outside.
\end{tabular}
\end{center}
This should definetly falls into the case of our assumption (2), and be classified as neutral, as even after swapping, the original specific $p$ still shows one potential scenario, given the original $h$. We assume, this arises from the creating process of \ac{SNLI}, as not many annotators seem to have added a lot more information, when creating the neutral hypothesis. Hence, due to the lackof similar neutral samples in the training, the model does not predict according to our assumption (2), if a huge amount of new information is added. Looking at the data that is re-predicted after spwapping as entailment or neutral, it seems, that those samples are in line with our assumptions (1), (2). We ignore the misclassifications resulting in contradiction and focus on those two labels instead, since they describe the scenarios we initally intended. In Figure \ref{fig:alignment_entailment_inversed} we visualize samples predicted as entailment (left) and neutral (right) after swapping.
\begin{figure}[tph!]	\centering
\includegraphics[totalheight=7cm]{fig/alignment_entailment_inversed.png}
	\caption{Visualization of samples predicted as entailment (left) and neutral (right) after swapping $p$ and $h$.}
	\label{fig:alignment_entailment_inversed}
\end{figure}
Indeed, samples that seemingly are paraphrased, based on the model's prediction, show highly identical meaning representations over all dimensions. Similarily, all samples that are now labelled as neutral and before swapping were considered entailment, show a hight tendency of encoding only a subset of information of the new hypothesis within the new premise. Both plots show, that our assumptions (1) and (2) are correct, accepting the fact that other scenarios exist, as shown by some swapped contradicting samples.
\subsubsection{Neutral and contradiction analysis}\label{sec:understanding_align_neutral_contr}
We have shown, that we indeed are able to observe, how the model identifies the entailing label, and how this differs from neutral, which in fact follows a very intuitive pattern. We also try to get more insights how neutral sentence-pairs  differ from contradiction based on the sentence representations, especially, w.r.t. assumtion (3). The results (without swapping) for 150 correctly classified neutral and contradicting examples respectively are displayed in Figure \ref{fig:neutr_contr_uninversed}.
\begin{figure}[tph!]	\centering
\includegraphics[totalheight=7cm]{fig/neutr_contr_uninversed.png}
	\caption{Visualitazion of 150 sentence pairs ($p$, $h$) correctly labelled as \textit{neutral} (left) and \textit{contradiction} (right).}
	\label{fig:neutr_contr_uninversed}
\end{figure} 
Unfortunately, we do not likewise find any patterns and also do not succeed by seperating them for a more detailed analysis into smaller subgroups. This does not mean that our assumption (3) is incorrect, since indeed contradicting samples show very distinct information in $p$ and $h$. Yet the same thing obviously happens for the neutral label. It  might be slightly less distinct information when looing at the absolute numbers, however this difference is far from being representative. We find an explanation for this issue by looking into the samples. Consider the following two samples, classified correctly as neutral:
\begin{center}
\begin{tabular}{lr}
\textbf{Premise} & A group of kite surfers are busy surfing some waves. \\
\textbf{Hypothesis} & The kite surfers are participating in a race. \\
\midrule
\textbf{Premise} & A baby laughing on the floor. \\
\textbf{Hypothesis} & A baby is being tickled.
\end{tabular}
\end{center}
In both cases, premise and hypothesis do contain distinct information, yet this information in the case of the neutral sentence-pair is not mutually exclusive but highly compatible with each other, and may also be true. We also look at correctly classified contradicting samples:
\begin{center}
\begin{tabular}{lr}
\textbf{Premise} & A boy eating at a table. \\
\textbf{Hypothesis} & A boy coloring at a table. \\
\midrule
\textbf{Premise} & A baby laughing on the floor. \\
\textbf{Hypothesis} & A toddler is crying.
\end{tabular}
\end{center}
Also in this case, both sentences encode different information, this time however it is not compatible, as the baby is either ``laughing'' or ``crying'' and the boy is either ``eating'' or ``coloring'', but not both things. We see that assumption (3) holds for contradiction in terms of recall, however it is not sufficient to distinguish between neutral and entailment. We conclude that in order to do so, one must identify, which dimensions can be true simultaneously, indicating neutral, and which dimensions exclude each other, indicating contradiction. Yet, to see how this is done within the model, we need to understand the \ac{MLP} on top of the sentence representation. This goes back to the main drawback of standard neural networks, being hard to interpret. We found, some useful insights, how the model leverages from the encoded sentence representations, following simple human intuitions. We do not further investigate in the differntiation of those dimensions and end our model understanding analysis at this point.

%\subsubsection{Experiments}
\subsection{Summarizing the insights on max-pooled sentence-representations}
We have shown that it is possible, to identify the general meaning of a dimension based on the words, that are responsible for the dimensions value. We observe that high values within dimensions encode the presence of information, which intuitively goes with the nature of the max-pooling mechanism, while low values indicate the absence of information. In our experiment, we additionally have shown, that knowing the encoding-scheme and information of a dimension, it is possible to change sentence-representations in meaningful and intended way, yet observed minor side-effects. In the second section, we showed that by aligning dimensions and knowing their encoded information, it is possible to understand sentence representations well enough, to interpret criteria, leaing to the prediction. In the last part we analysed, how the model actually performs the alignment between the sentence representations, and observed an intuitive explanation for several different types, how sentences relate to each other w.r.t. label.
\newline

\noindent
All these results are based on a relatively limited amount of rather small sentences, stemming from \ac{SNLI} and thus only show experimental results. For a general claim, similar experiments should be conducted on a larger scale, using more samples, containing longer text and using different models (all with max-pooled sentence representation) to see if those claims still holds. 
\subsection{Identification of missing knowledge}
In order to see, what information is not captured by the model, and can be helpful when integrated, we analyse the errors made on \ac{SNLI}. This analyses uses Shortcut-Stacked Encoder\textsuperscript{$\dagger\dagger$}, being closer to the reported accuracy by \cite{nie2017shortcut}. 
\subsubsection{Approach}
Aiming for missing knowledge, and not for a general error analysis, we focus on misclassified samples with gold label contradiction, predicted as entailment and vice versa, as we find that the label neutral was overall not well understood by the \ac{SNLI} annotators. To simplify the process we sample, according to this constraint, sentence-pairs having a high lexical overlap\footnote{Only considering a sentence pair, if at least 50\% of the words within the shorter sentence are contained in the other sentence, using a \ac{BoW} perspective. Casing is ignored.}, and look for the knowledge, required by a human to predict the correct label. In total, we look at 196 samples, incorrectly predicted as entailment, and 163 samples, incorrectly predicted as entailment, to identify common categories of missing knowledge. Note, that by pre-filtering the samples this way, our results exclude certain aspects. For instance many contradicting samples have multiple exclusive elements, resulting in a low lexical overlap.
\subsubsection{Results}
We show our results with the identified categories, incuding sample sentence-pairs for both labels in this section.

\paragraph*{Misclassified contradicting samples}
Table \ref{tab:misclassified_orig_contr} shows the misclassifications of samples, labelled incorrectly as entailment.
\begin{table}[tph!]
\centering
\begin{tabular}{cccl}
\textbf{Problem} & \textbf{Type} & \textbf{Amount} & \textbf{Example} \\
\toprule
cohyponym & Nouns & 29 & \specialcell{A \textit{creek} runs through the grassy area.\\A \textit{lake} appears in the grassy area.}\\
cohyponym & Verbs & 32 & \specialcell{The boy is \textit{riding} his skateboard.\\The boy is \textit{carrying} his skateboard with him.}\\
cohyponym & Amounts & 33 & \specialcell{\textit{Three} people resting on a snowy mountain.\\\textit{Four} people are on a snowy mountain.}\\
antonym & Adjectives & 13 & \specialcell{The ground is \textit{covered} in snow.\\The ground is \textit{visible}.}\\
antonym & Verbs & 14 & \specialcell{boy \textit{pushing} wagon with two pumpkins in it\\A boy is \textit{pulling} a wagon with two pumpkins in it.}\\
antonym & Prepositions & 17 & \specialcell{A man walking \textit{down} stairs.\\The man is walking \textit{up} the stairs.}\\
\midrule
structure & All & 15 & \specialcell{a sheep chases a dog.\\There is a dog chasing a sheep.}\\
common sense & All & 20 & \specialcell{Someone in a 3ft swimming pool.\\A person is in a very large and deep pool.}\\
\midrule
\textit{ignored} & \textit{ignored} & 23 & \specialcell{A man climbing a rock wall.\\A man climbs the wall.}\\
\midrule
\textbf{Total} & - & \textbf{196} & - \\
\bottomrule      
\end{tabular}
\caption{Misclassified samples with gold label \textit{contradiction}, predicted as \textit{entailment}.}
\label{tab:misclassified_orig_contr}
\end{table}
We find that most problems arise from words sharing lexical relations, namely antonomy or cohyponomy, for different kind of \ac{POS}. We opt to show ``amounts'' as a seperate category, due to its high frequency. While Verb antonyms may also be considered as cohyponyms, we also list them seperately, if they refer to the opposite meaning. Some samples, listed under ``structure'', require the model to take word-order into consideration, and is mostly represented by semantic role reversal. We assign all samples to ``common sense'', that require information, that cannot be retrieved using lexical semantic relations and usually need additional information, which is only \textit{implied} by the described entity or activity. Any sentence-pair, where we could not identify the required knowledge, due to not agreeing with the label or due to being highly ungrammatical, or some rare very specific details, are ignored in our results, categorized as ``ignored''.

\paragraph*{Misclassified entailing samples}
Analysing the required knowledge for misclassified entailing samples as contradtiong, we identify different categories, depicted in Table \ref{tab:misclassified_orig_ent}.
\begin{table}[tph!]
\centering
\begin{tabular}{cccl}
\textbf{Problem} & \textbf{Required knowledge} & \textbf{Amount} & \textbf{Example} \\
\toprule
Paraphrasing & lexical knowledge & 16 & \specialcell{Two people play \textit{foosball}.\\Two people are playing \textit{table soccer}.}\\
\specialcellc{Paraphrasing\\(negated opposite)} & lexical knowledge & 19 & \specialcell{A young boy is \textit{sleeping}.\\A child is \textit{not awake}.}\\
Paraphrasing & world knowledge & 20 & \specialcell{girl \textit{opening} cosmetics \textit{bottle}\\The girl is \textit{removing the top} off the \textit{bottle}.}\\
Generalization & lexical knowledge & 30 & \specialcell{The two \textit{boxers} are females.\\There are two female \textit{athletes}.}\\
Implication & world knowledge & 53 & \specialcell{A hockey player makes a shot.\\A hockey player \textit{is on ice}.}\\
\midrule
\textit{ignored} & \textit{ignored} & 25 & \specialcell{two people sit on a bench.\\two people sit on sand near water.} \\
\midrule
\textbf{Total} & - & \textbf{163} & - \\
\bottomrule      
\end{tabular}
\caption{Misclassified samples with gold label \textit{entailment}, predicted as \textit{contradiction}.}
\label{tab:misclassified_orig_ent}
\end{table}
Essentially, we find three different ways of how the relation between $p$ and $h$ can be described, and differentiate between lexical- and world-knowldge, being required for the correct prediction. In the example of paraphrasing, for ``foosball'' and ``table soccer'' or ``sleeping and ``awake'', it is sufficient to detect that they are synonyms or antonyms respectively. The third case requires some actual understanding of the process of ``opening'' in the context with a ``bottle''. While some of these samples may be understood by knowing several meronyms, we consider them to require a deeper conceptual understanding of how thing work, hence world knowledge. The largest group, ``implication'',   is interesting, as all $h$ contain additional information, not directly given by $p$. While this case should usually be labelled as neutral, in this case the additional information is automatically implied\footnote{For the given sample, we assume that Americans see \textit{hockey} in the sense of \textit{ice-hockey}.} (even though not textual) by $p$. This implication also requires other external knowledge than lexical relations.

\subsubsection{Conclusions}
We observe that especially the classification of contradicting samples can be improved by using lexical relations, which are available in WordNet, as described in Section §\ref{sec:wordnet}. Those may partially be also helpful for the problems, identified on the entailing samples. Yet, in this case, world-knowledge seems more relevant, as could be gained from resources like Wikipedia (see Section §\ref{sec:wikipedia}). While obviously the goal of \ac{NLI} is, to have proper reasonng capabilities, including dealing with world knowledge, we emphasize the aspect of lexical knowledge, arguing that any insights on how to incporporate this (simpler) information, may later be used to include world-knowledge.
\newline

\noindent
Note that the identified problems with lexical relations, especially \textit{antonomy} and \textit{cohyponomy} in Table \ref{tab:misclassified_orig_contr}, refer to the same problem, stemming from the nature of distributed representations. Essentially, they follow the distributional hypothesis, described by \cite{pantel2005inducing} as: 
\begin{quotation}
\textit{``[...] words that occur in the same contexts tend to have similar meaning.''}\citep{pantel2005inducing}
\end{quotation}
Considering a sentence like ``The president of Italy hopes to get re-elected.'', one can easily conclude that \textit{Italy} is a country based on its context. Even replacing \textit{Italy} by any made-up country would intuitively result in the same conclusion. Subsequently to being represented by their context words, distributed embeddings supposedly have better generalization abilities \citep{lecun2015deep} as modely may rely on the fact, that similar words are represented similarily. However also mutually exclusive co-hyponyms or antonyms often share similar contexts (like most countries could replace ``Italy'' in the given example), resulting in very similar vector representation despite opposite meanings in one aspect. This is a known problem \citep{sahlgren2008distributional} of distributed word representations and several approaches, as explained in Section §\ref{sec:embeddings_improvements_relwork}, aim to fix these problems in the embedding space.